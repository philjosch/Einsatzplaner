\section{Version 1.3}\label{versionshistorie:1:3}
\subsection{Version 1.3.0}
\label{version:1:3:0}
Veröffentlicht am 23.12.2017
\subsubsection{Neu}
\begin{itemize}
  \item
  Es kann jetzt eine Übersicht über die einzelnen Aktivitäten einer person ausgegeben werden, die Funktion dazu findet sich in der Einzelansicht des Personalfensters.
  \item
  Als weitere Kategorie für eine Aufgabe, kann jetzt noch "`Ausbildung"' angegeben werden.
  \item
  Es können nun auch zusätzliche Kilometer eingegeben werden, diese werden auf die automatisch berechneten Kilometer angerechnet.
\end{itemize}

\subsubsection{Verbessert}
Bei einer Reservierung kann jetzt eine zweite Teilstrecke eingegeben werden.

\subsubsection{Fehlerbehebungen}
\begin{itemize}
  \item
  Bei einer Änderung an einer Reservierung wird vor dem Schließen wieder nachgefragt, ob die Änderungen übernommen werden sollen.
  \item
  Optimierung beim Personalfenster, sodass es kleiner als bisher gemacht werden kann.
\end{itemize}


\subsection{Version 1.3.1}
\label{version:1:3:1}
Veröffentlicht am 14.03.2018
\subsubsection{Neu}
Das Arbeitseinsatzfenster wurde intelligenter. Es wählt eine bestimmte Kategorie für die Personaltabelle aus, wenn es dies aus dem Anlass folgern kann.

\subsubsection{Verbessert}
\begin{itemize}
  \item
  Wird personal für einen Fahrtag oder Arbeitseinsatz benötigt, wird dies farblich in der Einzelansicht kenntlich gemacht.
  \item
  Das Personal, dass bei einem Schnupperkurs eingetragen wurde, wird jetzt in der Listen- und Einzelansicht ausgegeben
  \item
  Externe Personen können nun auch mit "`Nachname, Vorname"' in Listen eingetragen werden.
\end{itemize}

\subsubsection{Fehlerbehebungen}
\begin{itemize}
  \item
  Probleme beim Drucken der Einzel- und Listenansicht zumindest unter macOS, wenn im Druckdialog "`als PDF speichern"' oder ähnliches gewählt wurde, sind behoben.
  \item
  Veränderungen in der Personaltabelle einer Aktivität wurde nicht korrekt übernommen.
  \item
  Fehler behoben, der es ermögliche nicht-Betriebsdienstpersonal als Zub einzutragen.
  \item
  Weitere Fehlerbehebungen, u.A. beim Darstellen der Daten nach dem Öffnen der Datei.
\end{itemize}
