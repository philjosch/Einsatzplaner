\section{Version 1.4}\label{versionshistorie:1:4}
\subsection{Version 1.4.0}
\label{version:1:4:0}
Veröffentlicht am 23.7.2018
\subsubsection{Neu}
\begin{itemize}
  \item
  Das Programm merkt sich die zuletzt verwendeten Dateien. Sie können unter „Datei > Zuletzt benutzt“ direkt geöffnet werden.
  \item
  Im Kalender können jetzt direkt Arbeitseinsätze an einem ausgewählten Tag erstellt werden.
\end{itemize}

\subsubsection{Verbessert}
Für eine Person kann jetzt auch noch eine zusätzliche Anzahl an Aktivitäten angegeben werden. Bisher war dies nur für Zeiten und Kilometer möglich.

\subsubsection{Fehlerbehebungen}
\begin{itemize}
  \item
  Die Druckausgabe wurde verbessert. Unter macOS ist nun das korrekte Erstellen einer PDF-Datei möglich.
  \item
  Mehrere Fehler beim Kalender wurden behoben.
  \item
  Weitere verschiedene Fehlerbehebungen und Verbesserungen, unter anderem bei der Verwaltung von Reservierungen.
\end{itemize}

\subsection{Version 1.4.1}
\label{version:1:4:1}
Veröffentlicht am 25.9.2018
\subsubsection{Neu}
Kalender zeigt jetzt den Anlass einer Aktivität und eines Fahrtages an, sofern es möglich ist.

\subsubsection{Verbessert}
\begin{itemize}
  \item
  Fahrtagfenster und Fenster für Aktivitäten optimiert, sodass sie den Platz besser nutzen.
  \item
  Verbesserter Export von Arbeitseinsätzen, sowohl in der Listen- als auch in der Einzelansicht.
  \item
  Fahrtage und Aktivitäten werden jetzt nicht mehr nur nach Datum sondern auch nach Beginn und Endzeit sortiert.
  \item
  Die Sitzplätze werden jetzt besser und schneller verteilt.
  \item
  Das Programm benötigt bei längerem Betrieb weniger Arbeitsspeicher.
  \item
  Code optimiert und kleinere Funktionen verbessert.
\end{itemize}

\subsubsection{Fehlerbehebungen}
\begin{itemize}
  \item
  Begleiter ohne betriebliche Aufgaben erschienen unter Umständen nicht in der Einzelansicht.
  \item
  Fahrtage und Aktivitäten können jetzt ohne Probleme gelöscht werden.
  \item
  Bei einem Arbeitseinsatz wird jetzt die Kategorie auch nach einem erneuten Öffnen intelligent bestimmt.
  \item
  Bei der Personalübersicht werden die Zeiten für die Ausbildung verlässlich angezeigt.
  \item
  Weitere Fehlerbehebungen zur Verbesserung der Stabilität
\end{itemize}
