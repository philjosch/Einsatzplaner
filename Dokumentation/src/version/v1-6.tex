\section{Version 1.6}\label{version:1:6}
\subsection{Version 1.6.0}
\label{version:1:6:0}
Veröffentlicht am 02.05.2020
\subsubsection{Neu}
\begin{itemize}
  \item
  Die Personalverwaltung wurde verbessert, indem jetzt alle Vereinsmitglieder aufgenommen und verwaltet werden können.
  Ebenso können verschiedene persönliche Daten und Kontaktdaten eingegeben und auch entsprechend exportiert werden.
  \item
  Eine Person kann jetzt mehrfach bei einem Arbeitseinsatz oder Fahrtag eingetragen werden,
  vorausgesetzt die Aufgabe ist jeweils verschieden.
  \item
  Anzeige der Auslastung der einzelnen Züge anhand der eingetragenen Reservierungen.
  \item
  Die Liste der Reservierungen kann nach Zügen gefiltert werden.
  \item
  Komplett überarbeitete Dokumentation.
\end{itemize}

\subsubsection{Verbessert}
\begin{itemize}
  \item
  Es gibt jetzt eine neue Kategorie "`Infrastruktur"'.
  Diese kann z.B.\ für Streckenarbeiten und Vegetationskontrollen genutzt werden.
  \item
  Die Anzahl der benötigten Lokführer kann beliebig zwischen null und zwei festgelegt werden,
  falls mit mehr als einem Triebfahrzeug gefahren wird.
  \item
  Die zusätzlichen Stunden und Mindeststunden können jetzt minutengenau eingegeben werden.
  \item
  Die Summe der Spalten in der Tabelle der Gesamtübersicht bezieht sich immer auf die aktuell angezeigten Personen.
  \item
  Beim Export von Daten können im Druckerdialog jetzt auch Seitenformat und Ausrichtung bestimmt werden.
  \item
  Unzählige Verbesserungen und Optimierungen "`unter der Haube"', um die Geschwindigkeit und den Speicherverbrauch zu optimieren.
\end{itemize}

\subsubsection{Fehlerbehebungen}
\begin{itemize}
  \item
  Fehler bei der Akzeptanz bestimmter Fahrstrecken einer Reservierung behoben.
  \item
  Die Einträge der Aktivitäten im Kalender bleiben nicht mehr markiert.
  \item
  Verschiedene kleinere Fehler behoben.
\end{itemize}


\subsection{Version 1.6.1}
\label{version:1:6:1}
Veröffentlicht am 06.08.2020
\subsubsection{Neu}
\begin{itemize}
  \item
  Beim Export der Mitglieder werden nur noch die Mitglieder ausgegeben, die aktuell auch angezeigt werden.
  \item
  Die Personaldaten werden im Personalfenster in einer Übersicht angezeigt.
  \item
  Eine Statistik zeigt unter anderem an, wie viele Mitglieder aktiv oder passiv sind.
  \item
  Aktive werden nur noch anhand der geleisteten Gesamtstunden bewertet.
  Die Markierung der nicht erbrachten Mindeststunden für andere Kategorien bleibt aber bestehen.
  \item
  Die Sortierung der Namen in der Personalübersicht kann zwischen "`Vorname Nachname"' und "`Nachname, Vorname"' eingestellt werden.
  \item
  Die Personaldaten können auch als Detailansicht ausgegeben werden, ohne die Liste für alle Mitglieder ausgeben zu müssen.
\end{itemize}

\subsubsection{Verbessert}
\begin{itemize}
  \item
  Beim Löschen von Personen wurde eine Sicherheitsabfrage eingefügt, um versehentliches Löschen von Personen zu vermeiden.
  \item
  Mail-Adressen, die mehreren Mitgliedern zugeordnet sind, führen nicht mehr dazu, dass evtl. mehrere Mails an diese Adresse gesendet werden.
  \item
  Bei der Eingabe von Namen werden führende und abschließende Leerzeichen ignoriert.
  \item
  Per Knopfdruck kann jetzt auch eine Mail an eine einzelne Person geschrieben werden.
\end{itemize}

\subsubsection{Fehlerbehebungen}
\begin{itemize}
  \item
  Die hochgeladene Datei wird wieder im Querformat dargestellt.
  \item
  Personal, dass der Tätigkeit "`Infrastruktur"' zugeordnet ist, wird wieder in der Listenansicht ausgegeben.
  \item
  Behebt einen Fehler, bei dem das Programm abstürzt, wenn bestimmte Personen von Aktivitäten gelöscht werden.
  \item
  Behebt einen Fehler, durch den die eingegebenen Verbindungsdaten zum EPL-Server nicht überprüft wurden.
\end{itemize}


\subsection{\neu{Version 1.6.2}}
\label{version:1:6:2}
Veröffentlicht am 16.10.2020
\subsubsection{Neu}
\begin{itemize}
  \item Die Mindeststunden werden nicht mehr für Mitglieder berechnet, die noch nicht volljährig sind.
  \item Schreibschutz für geöffnete Dateien eingefügt.
\end{itemize}

\subsubsection{Verbessert}
\begin{itemize}
  \item Vergangene Aktivitäten des gleichen Tages werden nicht mehr ausgegeben, wenn "`ab Heute"' bzw. "`ab Jetzt"' gewählt wird.
  \item Dialog der Exportfunktion überarbeitet
  \item Export der Daten verbessert: Gesonderte Hervorhebung, wenn Personal bei Aktivitäten benötigt wird, die in Kürze (10 Tage) anstehen.
  \item Die Standardeinstellungen für den Export auf Druckern wird auch beim Datei-Upload benutzt.
\end{itemize}

\subsubsection{Fehlerbehebungen}
\begin{itemize}
  \item Behebt mehrere Fehler durch die das Programm abstürzt, wenn eine Reservierung, ein Fahrtag oder ein Arbeitseinsatz gelöscht wird.
  \item Zeilenumbrüche werden wieder korrekt ausgegeben und angezeigt.
  \item Diverse Verbesserungen und Optimierungen
\end{itemize}
