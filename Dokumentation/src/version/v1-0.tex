\section{Version 1.0}\label{versionshistorie:1:0}
\subsection{Version 1.0.0}
\label{version:1:0:0}
Veröffentlicht am 6.10.2016
\subsubsection{Allgemeines}
\begin{itemize}
  \item
  Verwaltung von Arbeitseinsätzen und Fahrtagen
  \item
  Neues Aussehen für das Startfenster: Ein Kalender ermöglicht die schnelle und sichere Navigation zu den Aktivitäten
  \item
  Verwaltung von Personal mit dessen Ausbildung
  \item
  Automatische Unterscheidung zwischen Zub. und Beil.o.b.A. aufgrund der Ausbildung der Personen
  \item
  Die Tabellen sind kompakter gestaltet und somit ist mehr Platz für Inhalt
  \item
  Unter macOS: Unterstützung der "`öffnen mit"' Funktio
  \item
  Automatisches Suchen nach Updates im Internet
  \item
  Speichert die letzte Position des Hauptfensters
  \item
  Neues Icon und Logo, das auch auf schwarzem und weißem Hintergrund gut zu erkennen ist
  \item
  Behebung von Fehlern und Problemen
\end{itemize}

\subsubsection{Fahrtag}
\begin{itemize}
  \item
  Neue Export-Funktion: Hier werden jetzt nur Reservierungen ausgegeben, nach Wagen und Name sortiert.
  \item
  Es ist möglich zusätzliches personal einzutragen, sodass die Personen nicht in die vier Gruppen eingeteilt werden müssen
  \item
  Automatische Verteilung der Sitzplätze bei einer Nikolausfahrt
  \item
  Anzeige, zu wieviel Prozent die jeweiligen Klassen und der gesamte Zug belegt sind
\end{itemize}

\subsubsection{Personal}
\begin{itemize}
  \item
  Anzeige, welche person bei welchen Aktivitäten mitgeholfen hat
  \item
  Bestimmen der Einsatzzeiten des Personals für einzelne Arbeitsbereiche (Z.B. Werkstatt, Zugbegleitung)
\end{itemize}


\subsection{Version 1.0.1}
\label{version:1:0:1}
Veröffentlicht am 7.10.2016
\subsubsection{Fehlerbehebungen}
\begin{itemize}
  \item Problem beim öffnen von Dateien (Externe Personen wurden nicht geladen, Personen wurden nicht im Arbeitseinsatzfenster angezeigt)
  \item Problem bei Anzeige von Personen (Die Informationen zu den Aktivitäten wurden unter Umständen falsch dargestellt)
\end{itemize}


\subsection{Version 1.0.2}
\label{version:1:0:2}
Veröffentlicht am 13.11.2016
\begin{itemize}
  \item
  Die Wagenreihung kann jetzt auch mit Leerräumen eingegeben werden.
  \item
  Personen können jetzt auch mit "`Nachname, Vorname"' in die Listen eingegeben werden.
  \item
  Die Einträge werden jetzt in der Seitenleiste richtig sortiert. Auch werden die Daten in der richtigen Reihenfolge exportiert.
  \item
  Die Personalübersicht wurde flexibler gestaltet, auch können verschiedenen Spalten angezeigt und exportiert werden.
  \item
  Der Text wird jetzt bei Bedarf im Kalender umgebrochen.
  \item
  Wenn die Maus über dem Eintrag in der Seitenleiste verweilt, wird der Anlass der Veranstaltung angezeigt.
  \item
  Wichtige Fahrtage werden als solche in der Seitenleiste angezeigt.
  \item
  Reservierungen werden jetzt auf Plausibilität geprüft (z.B. kein Einstig am letzten Haltepunkt eines Zuges).
  \item
  Weitere Verbesserungen bei der Stabilität und Fehlerbehebungen.
\end{itemize}


\subsection{Version 1.0.3}
\label{version:1:0:3}
Veröffentlicht am 2.12.2016
\subsubsection{Verbesserungen}
\begin{itemize}
  \item
  Beim Export werden jetzt die Aufgaben der Personen angegeben, wenn sie unter sonstigem Personal gelistet wurden
  \item
  Arbeitseinsätze werden jetzt standardmäßig in der Listenansicht exportiert
  \item
  Bei Nikolausfahrten werden die Reservierungen nicht mehr auf dem Übersichtsblatt ausgegeben, sie müssen ab sofort über die Funktionen Reservierungen exportieren ausgegeben werden
\end{itemize}

\subsubsection{Fehlerbehebungen}
\begin{itemize}
  \item
  Personen können jetzt wieder in die Personallisten korrekt eingetragen und gelöscht werden, auch wenn eine Bemerkung angegeben ist
  \item
  Die Auswahl eines Datum bei der Export-Funktion für das "`Bis Datum"' funktioniert jetzt wieder wie erwartet
\end{itemize}
Weitere kleinere Verbesserungen und Fehlerbehebungen


\subsection{Version 1.0.4}
\label{version:1:0:4}
Veröffentlicht am 14.12.2016

Der Fehler, das das Programm beim Export der Personalübersicht bei manchen Benutzern abstürzt, wurde behoben.

Weitere kleine Fehlerkorrekturen und Verbesserungen.
