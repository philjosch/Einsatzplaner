\section{Version 1.5}\label{versionshistorie:1:5}
\subsection{Version 1.5.0}
\label{version:1:5:0}
Veröffentlicht am 31.3.2019
\subsubsection{Neu}
\begin{itemize}
  \item
  Der Einsatzplan kann als PDF direkt aus dem Programm auf einen vorher konfigurierten Webserver hochgeladen werden. Dies geschieht entweder beim Speichern oder manuell.
  \item
  Automatisches Sichern: Das Programm speichert bei Wunsch nach einer bestimmten Zeit automatisch eine Backup-Datei.
  \item
  Noch unbekannte Einsatzzeiten können jetzt als solche ausgewiesen werden.
  \item
  Fahrtage und Arbeitseinsätze können direkt aus dem jeweiligen Fenster heraus gelöscht werden.
\end{itemize}

\subsubsection{Verbessert}
\begin{itemize}
  \item
  In der Personalübersicht werden jetzt auch die Mindeststunden der Personen angezeigt.
  \item
  In der Listenansicht von Arbeitseinsätzen wurden redundante Informationen entfernt.
  \item
  In der Listenansicht von Arbeitseinsätzen wird jetzt auch ein etwaiger Ort angegeben.
  \item
  Eine Aktivität wird erst nach einer Sicherheitsabfrage gelöscht.
\end{itemize}

\subsubsection{Fehlerbehebungen}
\begin{itemize}
  \item
  Beim Eintragen von Personal für einen Arbeitseinsatz wurde die Person unter Umständen nicht richtig übernommen.
  \item
  Kleinere Optimierungen und Verbesserungen
\end{itemize}


\subsection{Version 1.5.1}
\label{version:1:5:1}
Veröffentlicht am 27.11.2019
\subsubsection{Neu}
Die Einsatzzeiten einer einzelnen Person können ab sofort als PDF gespeichert und gedruckt werden.

\subsubsection{Verbessert}
\begin{itemize}
  \item
  Der Export der Aktivitäten als Listen- und Einzelansicht wurde optimiert.
  \item
  Der Export der Personaldaten als Listen- und Einzelansicht wurde verbessert.
\end{itemize}

\subsubsection{Fehlerbehebungen}
\begin{itemize}
  \item
  Beim Eintragen von Personal für einen Arbeitseinsatz wurde die Person unter Umständen nicht richtig übernommen.
  \item
  Externes Personal eines Fahrtages wird jetzt wieder im entsprechenden Fenster dargestellt.
  \item
  Die Personalübersicht bleibt sortiert, auch wenn sie aktualisiert wird.
  \item
  Ein Problem beim automatischen Speichern wurde behoben.
  \item
  Die Einstellung des Zeitraums beim Datei-Upload wird jetzt zuverlässig verwendet.
  \item
  Kleinere Verbesserungen und Fehlerbehebungen
\end{itemize}


\subsection{Version 1.5.2}
\label{version:1:5:2}
Veröffentlicht am 22.3.2020
\subsubsection{Neu}
Durch einen Doppelklick auf eine Person in der Gesamtübersicht des Personalfensters wird die entsprechende Einzelansicht angezeigt.

\subsubsection{Verbessert}
Export der Personalübersichten verbessert, indem Stunden besser formatiert werden.

\subsubsection{Fehlerbehebungen}
\begin{itemize}
  \item
  Ein Fehler wurde behoben, bei dem das Programm beim Beenden abstürzt, wenn das automatische Speichern deaktiviert war.
  \item
  Beim Export der Personaldaten werden die Dateieinstellungen übernommen.
\end{itemize}
