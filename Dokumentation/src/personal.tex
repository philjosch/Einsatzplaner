\chapter{Personal}
\section{Einzelansicht}
In der linken Liste sind alle Personen gelistet, die im System registriert sind.
Rot markiert sind die Personen, die die notwendige Stundenzahl noch nicht geleistet haben.
(Siehe Gesamtansicht für weitere Informationen)


\paragraph{Stammdaten}
Die Angabe ``Entfernung'' bei einer Person dient dazu die gefahrene Wegstrecke zu berechnen.

Bei Ausbildung kann man die Ausbildung der Person angeben.
Wenn ein bestimmte Aufgabe eine Ausbildung verlangt, muss dies hier eingestellt werden.

Hinweis:
Eine Person kann erst dann gelöscht werden, wenn sie in keiner Aktivität mehr eingetragen ist.
So muss man zuerst die Person aus den Aktivitäten entfernen.


\paragraph{Aktivitäten}
Der Reiter zeigt eine Tabelle an, in der die Aktivitäten angezeigt werden, bei denen die ausgewählte Person mitgeholfen hat.


\paragraph{Konto}
Sie erhalten eine Übersicht über die Zeiten aufgeteilt nach den verschiedenen Zeit-Konten.
Ebenso können Sie hier zusätzliche Stunden und Kilometer eintragen, welche die Person geleistet hat, aber keiner spezifischen Aktivität zugeordnet wurden.
Die Anzahl der zusätzlichen Aktivitäten kann ebenfalls erfasst werden.
Ebenso kann man schnell sehen, in welchen Gebieten die Person ihre Arbeit geleistet hat und wieviele Mindeststunden Sie jeweils in den Gebieten erbringen muss.


Hinweis:
Für alle Darstellungen werden nur die Zeiten angerechnet, die bisher auch wirklich abgeleistet wurden, also deren Datum in der Vergangenheit lag.
Dies gilt auch für die Gesamtübersicht!


Mit den Knöpfen ``Als PDF speichern\dots'' und ``Drucken\dots'' kann eine Übersicht erstellt werden, die für jede Person eine individuelle Auflistung enthält, aus der ersichtlich wird, wer wieviele Stunden geleistet hat.
Dort wird auch angegeben, warum eine Person die Mindeststunden nicht erreicht hat.
Ebenso wird ein Blatt erstellt, das eine Übersicht über alle geleisteten Stunden enthält.



\section{Gesamtansicht}
In dieser Tabelle wird eine Übersicht über alle Personen gegeben.
Die Einfärbung erfolgt anhand der mindestens zu erbringenden Stunden.
Die Tabelle kann nach den verschiedenen Spalten sortiert werden, sodass man sich einen besseren Überblick verschaffen kann.
Ebenso befindet sich eine Zeile in der Tabelle, welche die Summe der jeweiligen Einträge angibt.
Diese Zeile wird bei der Ausgabe mit ausgegeben.

Mit Hilfe der Auswahlkästchen können die verschiedenen Spalten mit den Zeiten der Personen an- und ausgeschaltet werden.

Mit den Knöpfen ``Tabelle als PDF speichern\dots'' und ``Tabelle drucken\dots'' kann man die Daten der Tabelle exportieren.

Die Mindeststunden können über das Menü ``Personalmanagement'' \nach ``Mindeststunden\dots'' bearbeitet werden.
Standardmäßig sind 10 Stunden für alle eingestellt.
Lokführer müssen 100 Stunden als Lokführer abgeleistet haben.
Diese Werte können auch geändert werden.
In diesem Dialog können für jedes Konto die mindestens zu absolvierenden Zeiten angeben werden.

Die Mindeststunden für Ausbildung werden nur angewendet, wenn die Person eine bestehende betriebliche Ausbildung hat.
