\chapter{Allgemeines}\label{epl:allg}
Das EPL-Programmpaket dient dazu die Verwaltungstätigkeiten einer Museumseisenbahn zu verbessern.
Dazu können mit dem Programm \Einsatz Fahrtage und Arbeitseinsätze verwalten werden,
sowie die Einsatzzeiten der Mitglieder erfasst und berechnet werden.
Fahrtage unterschieden sich von Arbeitseinsätzen dadurch, dass das Personal direkt bestimmten Aufgabengebieten zugeordnet werden kann.
Ebenso können Reservierungen eingetragen werden.
Mit dem Programm \Personal kann eine Verwaltung der Mitglieder und Speicherung der entsprechenden Daten durchgeführt werden.

\begin{figure}[h]
  \centering
  \begin{subfigure}{0.3\textwidth}
    \includegraphics[width=\textwidth]{../../Icon/EPL.png}
    \caption{Logo von \EPL}
  \end{subfigure}

  \begin{subfigure}{0.3\textwidth}
    \includegraphics[width=\textwidth]{../../Icon/Einsatzplaner.png}
    \caption{Logo des \Einsatz}
  \end{subfigure}
  \begin{subfigure}{0.3\textwidth}
    \includegraphics[width=\textwidth]{../../Icon/Personalplaner.png}
    \caption{Logo des \Personal}
  \end{subfigure}
  \caption{Die Logos des Programmpakets}
\end{figure}


\section{Die EPL-Datei}\label{epl:allg:datei}

\subsection{Schreibschutz}\label{epl:allg:datei:schreibschutz}
Beim Öffnen einer Datei \datei{X.ako} wird automatisch eine neue Datei mit dem Namen \datei{X.ako.lock} erstellt.
Durch diese Datei wird sichergestellt, dass nicht mehrere Personen gleichzeitig an der Datei \datei{X.ako} arbeiten.
Wird eine bereits geöffnete Datei erneut geöffnet (vom gleichen oder fremden Benutzer),
kann sie nur schreibgeschützt geöffnet werden und Änderungen können nicht gespeichert werden.
Beim Schließen einer Datei bzw.\ beim Beenden des Programms wird die Datei \datei{X.ako.lock} automatisch wieder gelöscht.
Danach kann ein anderer Benutzer die Datei wieder uneingeschränkt öffnen.

\begin{hinweis}
  Der Schreibschutz steht nur effektiv zur Verfügung, wenn alle Beteiligten Version 1.6.2 oder neuer des Einsatzplaners verwendet.
  Wird eine frühere Version verwendet, kann die Datei allerdings weiterhin uneingeschränkt geöffnet und bearbeitet werden!
\end{hinweis}


\begin{hinweis}
  Die oben beschriebene Datei sollte unter normalen Umständen nicht manuell gelöscht werden, da dies zu Datenverlusten führen kann.
  Löschen Sie die Datei nur, wenn Sie sicherstellen können, dass keine weiteren Benutzer auf die Datei zugreifen.
\end{hinweis}


\begin{neu}
\subsection{Passwort-Schutz}\label{epl:allg:datei:passwort}
Es besteht die Möglichkeit die EPL-Datei mit einem Passwort-Schutz zu speichern.
Durch diesen können nur berechtigte Personen auf die Daten zugreifen.
Die Datei wird in diesem Fall verschlüsselt gespeichert.

\paragraph{Schutz einrichten und aufheben}
Der Schutz kann über den Eintrag \aktion{Eigenschaften} im Menü \aktion{Datei} geändert werden (Siehe auch \cref{fig:einsatz:kalender:upload}).
Das bisherige Passwort eingeben (sofern vorhanden) und das neue Passwort zweimal eingeben, um Tippfehler zu vermeiden.
Dann über den Knopf \button{Passwort ändern} die Änderung vornehmen.
Um einen vorhandenen Passwortschutz zu entfernen,
die beiden Felder für das neue Passwort frei lassen und mit dem Knopf bestätigen.


\begin{hinweis}
  Ohne das Passwort kann die Datei nicht mehr geöffnet werden und die Daten sind verloren.
  Stellen Sie deshalb sicher,
  dass Sie sich das Passwort merken können oder speichern es in einem Passwort-Manager.
\end{hinweis}

\end{neu}

\section{Installation}\label{epl:allg:installation}
\subsection{macOS}
Laden Sie die bereitgestellte Datei herunter und entpacken Sie die \datei{.dmg}-Datei durch einen Doppelklick.
Ziehen Sie den Ordner \datei{EPL} zum Installieren in Ihren Programmordner.
Alternativ können Sie die beiden Programm im Ordner \datei{EPL} auch einzeln in den Programmordner Ihres Systems bewegen.

\subsection{Windows}
\begin{itemize}
  \item Laden Sie den Installer für Windows herunter.
  Öffnen Sie das Programm, und folgen dem Installationsprogramm.
  \item Merken Sie sich den Pfad, an dem Sie den Einsatzplaner installieren, z.B.\ \datei{C:{\bslash}Program Files (x86){\bslash}Einsatzplaner}.
  \item Wenn Sie den Einsatzplaner das erste Mal installieren,
  bzw.\ beim Start eine Fehlermeldung erhalten,
  navigieren Sie zu dem notierten Ordner.
  \item Dort finden Sie eine Datei \datei{vc\_redist.x64}.
  Öffnen Sie diese und führen die Installation von \enquote{Microsoft Visual C++ Redistributable} durch.
  Hierdurch werden die fehlenden Dateien auf Ihrem Computer ergänzt.
  \item Die Programme können jetzt verwendet werden.
\end{itemize}



\section{Über das Dokument}\label{epl:allg:sonstiges}
Die Bilder in dieser Dokumentation stammen von Bildern der Entwicklerversionen des Programms für Version
1.7.1
und aus früheren Versionen.
Dadurch können Programmobjekte und Funktionen leicht von denen in der veröffentlichten Version abweichen.

Eine aktuelle Version des Programms und weitere Informationen finden Sie der Webseite \url{http://epl.philipp-schepper.de}
oder im Repository unter \url{https://github.com/philjosch/Einsatzplaner}.
