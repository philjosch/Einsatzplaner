\chapter{Versionshistorie}
\section{Version 1.0}
\subsection{Version 1.0.0}
Veröffentlicht am 6.10.2016
\subsubsection{Allgemeines}
\begin{itemize}
  \item
  Verwaltung von Arbeitseinsätzen und Fahrtagen
  \item
  Neues Aussehen für das Startfenster: Ein Kalender ermöglicht die schnelle und sichere Navigation zu den Aktivitäten
  \item
  Verwaltung von Personal mit dessen Ausbildung
  \item
  Automatische Unterscheidung zwischen Zub. und Beil.o.b.A. aufgrund der Ausbildung der Personen
  \item
  Die Tabellen sind kompakter gestaltet und somit ist mehr Platz für Inhalt
  \item
  Unter macOS: Unterstützung der "öffnen mit" Funktio
  \item
  Automatisches Suchen nach Updates im Internet
  \item
  Speichert die letzte Position des Hauptfensters
  \item
  Neues Icon und Logo, das auch auf schwarzem und weißem Hintergrund gut zu erkennen ist
  \item
  Behebung von Fehlern und Problemen
\end{itemize}

\subsubsection{Fahrtag}
\begin{itemize}
  \item
  Neue Export-Funktion: Hier werden jetzt nur Reservierungen ausgegeben, nach Wagen und Name sortiert.
  \item
  Es ist möglich zusätzliches personal einzutragen, sodass die Personen nicht in die vier Gruppen eingeteilt werden müssen
  \item
  Automatische Verteilung der Sitzplätze bei einer Nikolausfahrt
  \item
  Anzeige, zu wieviel Prozent die jeweiligen Klassen und der gesamte Zug belegt sind
\end{itemize}

\subsubsection{Personal}
\begin{itemize}
  \item
  Anzeige, welche person bei welchen Aktivitäten mitgeholfen hat
  \item
  Bestimmen der Einsatzzeiten des Personals für einzelne Arbeitsbereiche (Z.B. Werkstatt, Zugbegleitung)
\end{itemize}


\subsection{Version 1.0.1}
Veröffentlicht am 7.10.2016
\subsubsection{Fehlerbehebungen}
\begin{itemize}
  \item Problem beim öffnen von Dateien (Externe Personen wurden nicht geladen, Personen wurden nicht im Arbeitseinsatzfenster angezeigt)
  \item Problem bei Anzeige von Personen (Die Informationen zu den Aktivitäten wurden unter Umständen falsch dargestellt)
\end{itemize}


\subsection{Version 1.0.2}
Veröffentlicht am 13.11.2016
\begin{itemize}
  \item
  Die Wagenreihung kann jetzt auch mit Leerräumen eingegeben werden.
  \item
  Personen können jetzt auch mit "Nachname, Vorname" in die Listen eingegeben werden.
  \item
  Die Einträge werden jetzt in der Seitenleiste richtig sortiert. Auch werden die Daten in der richtigen Reihenfolge exportiert.
  \item
  Die Personalübersicht wurde flexibler gestaltet, auch können verschiedenen Spalten angezeigt und exportiert werden.
  \item
  Der Text wird jetzt bei Bedarf im Kalender umgebrochen.
  \item
  Wenn die Maus über dem Eintrag in der Seitenleiste verweilt, wird der Anlass der Veranstaltung angezeigt.
  \item
  Wichtige Fahrtage werden als solche in der Seitenleiste angezeigt.
  \item
  Reservierungen werden jetzt auf Plausibilität geprüft (z.B. kein Einstig am letzten Haltepunkt eines Zuges).
  \item
  Weitere Verbesserungen bei der Stabilität und Fehlerbehebungen.
\end{itemize}


\subsection{Version 1.0.3}
Veröffentlicht am 2.12.2016
\subsubsection{Verbesserungen}
\begin{itemize}
  \item
  Beim Export werden jetzt die Aufgaben der Personen angegeben, wenn sie unter sonstigem Personal gelistet wurden
  \item
  Arbeitseinsätze werden jetzt standardmäßig in der Listenansicht exportiert
  \item
  Bei Nikolausfahrten werden die Reservierungen nicht mehr auf dem Übersichtsblatt ausgegeben, sie müssen ab sofort über die Funktionen Reservierungen exportieren ausgegeben werden
\end{itemize}

\subsubsection{Fehlerbehebungen}
\begin{itemize}
  \item
  Personen können jetzt wieder in die Personallisten korrekt eingetragen und gelöscht werden, auch wenn eine Bemerkung angegeben ist
  \item
  Die Auswahl eines Datum bei der Export-Funktion für das "Bis Datum" funktioniert jetzt wieder wie erwartet
\end{itemize}
Weitere kleinere Verbesserungen und Fehlerbehebungen


\subsection{Version 1.0.4}
Veröffentlicht am 14.12.2016

Der Fehler, das das Programm beim Export der Personalübersicht bei manchen Benutzern abstürzt, wurde behoben.

Weitere kleine Fehlerkorrekturen und Verbesserungen.



\section{Version 1.1}
\subsection{Version 1.1.0}
Veröffentlicht am 21.12.2016
\subsubsection{Veränderungen}
\begin{itemize}
  \item
  Für jede Person kann eine zusätzliche Stundenanzahl für jedes Konto angegeben werden. Diese Zeiten können in der Einzelansicht einer person angegeben werden
  \item
  Es muss kein Haken mehr gesetzt werden, wenn die Reservierungen automatisch verteilt werden sollen
  \item
  Beim Export von Reservierungen wird jetzt auch der Zu- und Ausstieg angegeben
\end{itemize}

\subsubsection{Fehlerbehebungen}
\begin{itemize}
  \item
  Die Aufgabe einer Person bei einer bestimmten Aktivität wird jetzt korrekt gespeichert und geladen
  \item
  Weitere kleinere Fehlerbehebungen und Verbesserungen
\end{itemize}



\section{Version 1.2}
\subsection{Version 1.2.0}
Veröffentlicht am 10.10.2017
\subsubsection{Neu}
\begin{itemize}
  \item
  Die Mindeststunden können eingestellt werden, dies ist im Personalfenster möglich. Dort findet sich ein neuer Knopf mit der Bezeichnung Mindeststuden.
  \item
  Anzeige der Summe der Einsatzzeiten in der Personalübersicht und in der exportierten Tabelle. Diese Funktion ist immer aktiv und zeigt die Summe der Zeiten der jeweiligen Spalte an.
  \item
  Beim Export von Daten wird jetzt auch die Uhrzeit mit ausgegeben.
\end{itemize}

\subsubsection{Verbessert}
\begin{itemize}
  \item
  Kennzeichnung von externen Personen auf Triebfahrzeugen oder ähnlichem wurde vereinfacht und verbessert. Externe Personen können jetzt mit folgenden Begriffen gekennzeichnet werden, sodass sie als extern angesehen werden und nicht im Personalverzeichnis gesucht werden:
  \begin{itemize}
    \item „Extern“
    \item „Führerstand“
    \item „FS“
    \item „Schnupperkurs“
    \item „ELF“
    \item „Ehrenlokführer“
    \item „ELF-Kurs“
  \end{itemize}
  Hierbei ist es ausreichend, wenn einer dieser Begriffe in der Bemerkung vorkommt.
  \item
  Für registriertes Personal wurde eine ähnliche Funktion eingeführt, die verhindert, dass überprüft wird, ob eine Person die Qualifikation hat. Dies kann dann insbesondere für Ausbildungszwecke verwendet werden. Hier die Schlüsselbegriffe:
  \begin{itemize}
    \item   „Azubi“
    \item  „Ausbildung“
    \item  „Tf-Ausbildung“,
    \item  „Zf-Ausbildung“
    \item  „Tf-Unterricht“
    \item  „Zf-Unterricht“
    \item  „Weiterbildung“
  \end{itemize}
  \item
  In der Personalübersicht eines Arbeitseinsatzes und Fahrtages können jetzt die Aufgaben aus der voreingestellten Liste ausgewählt werden. Ebenso wurde ein extra Feld für die Bemerkung eingeführt. Die Zeiten können jetzt besser eingestellt werden.
  \item
  Die Sitzplätze bei Reservierungen können jetzt besser eingegeben werden. Hier wurde ein Fehler behoben.
\end{itemize}

\subsubsection{Fehlerbehebungen}
\begin{itemize}
  \item
  Beim öffnen einer Datei, wird jetzt wieder zu dem Monat gesprungen, an dem die Datei gespeichert wurde.
  \item
  Beim Öffnen eines Fahrtags wird die Bemerkung wieder geladen.
  \item
  Rechtschreibfehler korrigiert
  \item
  Weitere kleine Fehlerbehebungen und Verbesserungen
\end{itemize}



\section{Version 1.3}
\subsection{Version 1.3.0}
Veröffentlicht am 23.12.2017
\subsubsection{Neu}
\begin{itemize}
  \item
  Es kann jetzt eine Übersicht über die einzelnen Aktivitäten einer person ausgegeben werden, die Funktion dazu findet sich in der Einzelansicht des Personalfensters.
  \item
  Als weitere Kategorie für eine Aufgabe, kann jetzt noch "Ausbildung" angegeben werden.
  \item
  Es können nun auch zusätzliche Kilometer eingegeben werden, diese werden auf die automatisch berechneten Kilometer angerechnet.
\end{itemize}

\subsubsection{Verbessert}
Bei einer Reservierung kann jetzt eine zweite Teilstrecke eingegeben werden.

\subsubsection{Fehlerbehebungen}
\begin{itemize}
  \item
  Bei einer Änderung an einer Reservierung wird vor dem Schließen wieder nachgefragt, ob die Änderungen übernommen werden sollen.
  \item
  Optimierung beim Personalfenster, sodass es kleiner als bisher gemacht werden kann.
\end{itemize}


\subsection{Version 1.3.1}
Veröffentlicht am 14.03.2018
\subsubsection{Neu}
Das Arbeitseinsatzfenster wurde intelligenter. Es wählt eine bestimmte Kategorie für die Personaltabelle aus, wenn es dies aus dem Anlass folgern kann.

\subsubsection{Verbessert}
\begin{itemize}
  \item
  Wird personal für einen Fahrtag oder Arbeitseinsatz benötigt, wird dies farblich in der Einzelansicht kenntlich gemacht.
  \item
  Das Personal, dass bei einem Schnupperkurs eingetragen wurde, wird jetzt in der Listen- und Einzelansicht ausgegeben
  \item
  Externe Personen können nun auch mit "Nachname, Vorname" in Listen eingetragen werden.
\end{itemize}

\subsubsection{Fehlerbehebungen}
\begin{itemize}
  \item
  Probleme beim Drucken der Einzel- und Listenansicht zumindest unter macOS, wenn im Druckdialog "als PDF speichern" oder ähnliches gewählt wurde, sind behoben.
  \item
  Veränderungen in der Personaltabelle einer Aktivität wurde nicht korrekt übernommen.
  \item
  Fehler behoben, der es ermögliche nicht-Betriebsdienstpersonal als Zub einzutragen.
  \item
  Weitere Fehlerbehebungen, u.A. beim Darstellen der Daten nach dem Öffnen der Datei.
\end{itemize}



\section{Version 1.4}
\subsection{Version 1.4.0}
Veröffentlicht am 23.7.2018
\subsubsection{Neu}
\begin{itemize}
  \item
  Das Programm merkt sich die zuletzt verwendeten Dateien. Sie können unter „Datei > Zuletzt benutzt“ direkt geöffnet werden.
  \item
  Im Kalender können jetzt direkt Arbeitseinsätze an einem ausgewählten Tag erstellt werden.
\end{itemize}

\subsubsection{Verbessert}
Für eine Person kann jetzt auch noch eine zusätzliche Anzahl an Aktivitäten angegeben werden. Bisher war dies nur für Zeiten und Kilometer möglich.

\subsubsection{Fehlerbehebungen}
\begin{itemize}
  \item
  Die Druckausgabe wurde verbessert. Unter macOS ist nun das korrekte Erstellen einer PDF-Datei möglich.
  \item
  Mehrere Fehler beim Kalender wurden behoben.
  \item
  Weitere verschiedene Fehlerbehebungen und Verbesserungen, unter anderem bei der Verwaltung von Reservierungen.
\end{itemize}

\subsection{Version 1.4.1}
Veröffentlicht am 25.9.2018
\subsubsection{Neu}
Kalender zeigt jetzt den Anlass einer Aktivität und eines Fahrtages an, sofern es möglich ist.

\subsubsection{Verbessert}
\begin{itemize}
  \item
  Fahrtagfenster und Fenster für Aktivitäten optimiert, sodass sie den Platz besser nutzen.
  \item
  Verbesserter Export von Arbeitseinsätzen, sowohl in der Listen- als auch in der Einzelansicht.
  \item
  Fahrtage und Aktivitäten werden jetzt nicht mehr nur nach Datum sondern auch nach Beginn und Endzeit sortiert.
  \item
  Die Sitzplätze werden jetzt besser und schneller verteilt.
  \item
  Das Programm benötigt bei längerem Betrieb weniger Arbeitsspeicher.
  \item
  Code optimiert und kleinere Funktionen verbessert.
\end{itemize}

\subsubsection{Fehlerbehebungen}
\begin{itemize}
  \item
  Begleiter ohne betriebliche Aufgaben erschienen unter Umständen nicht in der Einzelansicht.
  \item
  Fahrtage und Aktivitäten können jetzt ohne Probleme gelöscht werden.
  \item
  Bei einem Arbeitseinsatz wird jetzt die Kategorie auch nach einem erneuten Öffnen intelligent bestimmt.
  \item
  Bei der Personalübersicht werden die Zeiten für die Ausbildung verlässlich angezeigt.
  \item
  Weitere Fehlerbehebungen zur Verbesserung der Stabilität
\end{itemize}



\section{Version 1.5}
\subsection{Version 1.5.0}
Veröffentlicht am 31.3.2019
\subsection{Neu}
\begin{itemize}
  \item
  Der Einsatzplan kann als PDF direkt aus dem Programm auf einen vorher konfigurierten Webserver hochgeladen werden. Dies geschieht entweder beim Speichern oder manuell.
  \item
  Automatisches Sichern: Das Programm speichert bei Wunsch nach einer bestimmten Zeit automatisch eine Backup-Datei.
  \item
  Noch unbekannte Einsatzzeiten können jetzt als solche ausgewiesen werden.
  \item
  Fahrtage und Arbeitseinsätze können direkt aus dem jeweiligen Fenster heraus gelöscht werden.
\end{itemize}

\subsubsection{Verbessert}
\begin{itemize}
  \item
  In der Personalübersicht werden jetzt auch die Mindeststunden der Personen angezeigt.
  \item
  In der Listenansicht von Arbeitseinsätzen wurden redundante Informationen entfernt.
  \item
  In der Listenansicht von Arbeitseinsätzen wird jetzt auch ein etwaiger Ort angegeben.
  \item
  Eine Aktivität wird erst nach einer Sicherheitsabfrage gelöscht.
\end{itemize}

\subsubsection{Fehlerbehebungen}
\begin{itemize}
  \item
  Beim Eintragen von Personal für einen Arbeitseinsatz wurde die Person unter Umständen nicht richtig übernommen.
  \item
  Kleinere Optimierungen und Verbesserungen
\end{itemize}


\subsection{Version 1.5.1}
Veröffentlicht am 27.11.2019
\subsubsection{Neu}
Die Einsatzzeiten einer einzelnen Person können ab sofort als PDF gespeichert und gedruckt werden.

\subsubsection{Verbessert}
\begin{itemize}
  \item
  Der Export der Aktivitäten als Listen- und Einzelansicht wurde optimiert.
  \item
  Der Export der Personaldaten als Listen- und Einzelansicht wurde verbessert.
\end{itemize}

\subsubsection{Fehlerbehebungen}
\begin{itemize}
  \item
  Beim Eintragen von Personal für einen Arbeitseinsatz wurde die Person unter Umständen nicht richtig übernommen.
  \item
  Externes Personal eines Fahrtages wird jetzt wieder im entsprechenden Fenster dargestellt.
  \item
  Die Personalübersicht bleibt sortiert, auch wenn sie aktualisiert wird.
  \item
  Ein Problem beim automatischen Speichern wurde behoben.
  \item
  Die Einstellung des Zeitraums beim Datei-Upload wird jetzt zuverlässig verwendet.
  \item
  Kleinere Verbesserungen und Fehlerbehebungen
\end{itemize}


\subsection{Version 1.5.2}
Veröffentlicht am 22.3.2020
\subsubsection{Neu}
Durch einen Doppelklick auf eine Person in der Gesamtübersicht des Personalfensters wird die entsprechende Einzelansicht angezeigt.

\subsubsection{Verbessert}
Export der Personalübersichten verbessert, indem Stunden besser formatiert werden.

\subsubsection{Fehlerbehebungen}
\begin{itemize}
  \item
  Ein Fehler wurde behoben, bei dem das Programm beim Beenden abstürzt, wenn das automatische Speichern deaktiviert war.
  \item
  Beim Export der Personaldaten werden die Dateieinstellungen übernommen.
\end{itemize}



\section{Version 1.6}
\subsection{Version 1.6.0}
Veröffentlicht am XX.YY.2020
\subsubsection{Neu}
\subsubsection{Verbessert}
\subsubsection{Fehlerbehebungen}
