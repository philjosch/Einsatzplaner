\documentclass[a4paper,ngerman,oneside]{scrbook}
\usepackage[ngerman]{babel}
\usepackage[utf8]{inputenc}
\usepackage[T1]{fontenc}
\usepackage{xspace}
\usepackage{hyperref}

\usepackage{scrhack}

\usepackage{enumitem}
\setlist[itemize]{noitemsep, topsep=0pt}

\usepackage{graphicx,tabularx}
\usepackage{textcomp}
\usepackage{xcolor}
\usepackage{subcaption}

\newcommand{\nach}{\xspace\textrightarrow\xspace}
\newcommand{\cmnd}[1]{\texttt{#1}\xspace}
\newcommand{\hinweis}[1]{\emph{\textbf{Hinweis:} #1}}
\newcommand{\neu}[1]{{\color{red} #1}}

\graphicspath{{./img/}}

\titlehead{}

\subject{\centering\includegraphics[width=0.25\textwidth]{../../Icon/EPL.png}}
\title{Anleitung EPL-Programmpaket}
\subtitle{Version 1.6.2}
\author{Philipp Schepper}
\date{Stand: \today}
\publishers{\Huge{\textsf{Arbeitsdatei}}}

\begin{document}
\maketitle

\frontmatter
\tableofcontents
\chapter{Änderungshistorie}
\begin{tabularx}{\textwidth}{l|X}
  Datum & Änderung \\
  \hline
  \hline
  02.05.2020 &
    Dokument für Version 1.6 komplett neu herausgegeben.\newline
    Wichtige Informationen zur Aktualisierung von Version 1.5 auf 1.6 in Kapitel~\ref{sonstiges:1.5-1.6}.\\
  \hline
  06.08.2020 &
    Version 1.6.1 eingefügt.\newline
    Veränderungen hauptsächlich in Kapitel~\ref{personal}.
    Ansonsten überwiegend Fehlerbehebungen.
    \\
  \hline
  XX.XX.2020 &
    Version 1.6.2 eingefügt.\newline
    Kapitel~\ref{schreibschutz} eingefügt.
\end{tabularx}


\mainmatter
\part*{EPL-Programmfamilie}
\chapter{Allgemeines}
Das Programm dienst dazu Fahrtage und Arbeitseinsätze einer Museumseisenbahn zu verwalten.

Fahrtage unterschieden sich von Arbeitseinsätzen dahingehend, dass das Personal direkt bestimmten Aufgabengebieten zugeordnet werden kann. Ebenso können bei Fahrtagen Reservierungen verwaltet werden.


\part{Einsatzplaner}
\chapter{Einsatzplaner}\label{einsatz:kalender}
Das Programm \Einsatz dient der Verwaltung der Arbeitseinsätze und der Fahrtage.
Der in \cref{fig:einsatz:kalender} gezeigte Kalender ist das Startfenster von \Einsatz.
In ihm werden die Fahrtage und Aktivitäten des ausgewählten Monats angezeigt.
Ebenso enthält eine Liste auf der rechten Seite alle Aktivitäten nach Datum sortiert.
Jeder Eintrag kann durch einen Doppelklick geöffnet werden.


Fahrtage und Arbeitseinsätze werden,
um einen schnelleren Überblick zu bekommen,
in verschiedenen Farben eingefärbt.
\begin{neu}
Das aktuelle Datum wird farblich hervorgehoben,
um eine schnellere Navigation zu erlauben.
\end{neu}

\begin{figure}[h]
  \includegraphics[width=\textwidth]{img/kalender}
  \caption{
  Startfenster des \Einsatz mit dem Kalender und den eingetragenen Fahrtagen und Arbeitseinsätzen}
  \label{fig:einsatz:kalender}
\end{figure}



\section{Aktivitäten verwalten}
\label{einsatz:kalender:anlegen}
\label{einsatz:kalender:löschen}
\paragraph{Fahrtag erstellen}
Um einen neuen Fahrtag anzulegen, klicken Sie auf den Knopf \button{Neuer Fahrtag}.
Es wird ein Fahrtag mit dem aktuellen Datum angelegt und das entsprechende Fenster geöffnet (siehe \cref{einsatz:fahrtag})

\paragraph{Arbeitseinsatz erstellen}
Ein neuer Arbeitseinsatz kann mit Hilfe des Knopfes \button{Neuer Arbeitseinsatz} angelegt werden.
Ein Arbeitseinsatz kann auch direkt im Kalender mit einen Klick auf \button{+} mit dem entsprechenden Datum erstellt werden.
In beiden Fällen öffnet sich danach das Fenster für den Arbeitseinsatz
(siehe \cref{einsatz:arbeitseinsatz}).




\paragraph{Löschen von Aktivitäten}
Um eine Aktivität zu löschen, wählen Sie diese in der Liste aus
und klicken auf den Knopf \button{Listeneintrag löschen}.
Nach einer Sicherheitsabfrage wird der Eintrag gelöscht und verschwindet dann auch aus dem Kalender.



\section{Navigation im Kalender}\label{einsatz:kalender:navigieren}
Mit den Knöpfen \button{Zurück}, \button{Vor} und \button{Heute} können Sie durch den Kalender navigieren.
Ebenso können sie über das Feld auch direkt ein Datum eingeben.



\section{Das Datei-Menü}
Im Datei-Menü stehen Ihnen folgende Funktionen zur Verfügung:
\begin{description}
  \item[Neu]
  Erstellt ein neues Fenster mit einer leeren Einsatzplaner-Instanz.

  \item[Öffnen \dots]
  Es wird ein Dialog geöffnet, mit dem eine \datei{.ako}-Datei geöffnet werden kann.

  \item[Zuletzt benutzt]
  Unter diesem Eintrag finden Sie die fünf zuletzt verwendeten Dateien.
  Sie können direkt geöffnet werden.
  Ebenso können Sie bei Bedarf die Liste leeren.

  \item[Speichern]
  Die Datei wird an dem bisher bekannten Pfad gesichert, oder Sie werden nach einem Ort zum Sichern der Datei gefragt.

  \item[Sichern unter \dots]
  Sie können einen Ort auswählen, an dem die Datei gespeichert werden soll.
  Zur automatischen Speicherung der Daten finden Sie in \cref{epl:allg:einstellungen} weitere Informationen.

  \item[Stammdaten sichern unter \dots]
  Diese Funktion bietet die Möglichkeit, die unveränderlichen Personaldaten zu exportieren.
  Ebenso werden die Datei-Einstellungen übernommen.
  Es werden somit keine Fahrtage oder Arbeitseinsätze gespeichert.
  Diese Funktion dupliziert sozusagen die aktuelle Datei und löscht dabei alle Fahrtage und Arbeitseinsätze.


  \item[Eigenschaften]
  Hier können Sie das Online-Tool aktivieren und konfigurieren.
  Weitere Informationen in \cref{einsatz:kalender:upload}.

  \item[Schließen]
  Schließt das aktuelle Fenster.
  Bei ungesicherten Veränderungen wird vor dem Schließen nachgefragt, ob die Änderungen gespeichert werden sollen.
  Wenn kein Fenster mehr geöffnet ist, wird das Programm beendet.

  \item[Export \dots]
  Ruft die Exportfunktion auf.
  Weitere Funktionen in \cref{einsatz:kalender:export}.
\end{description}


\section{Export}\label{einsatz:kalender:export}
\begin{figure}[!h]
  \centering
	\includegraphics[width=.5\textwidth]{img/export}
	\caption{Der Exportdialog im Hauptfenster.}
	\label{fig:einsatz:kalender:export}
\end{figure}
Über den Knopf \button{Export} öffnet sich der Dialog in \cref{fig:einsatz:kalender:export}.
Diese Funktion ist auch über den Tastaturbefehl \cmnd{cmd+P} beziehungsweise \cmnd{ctrl+P} zu erreichen.

Dort kann mit den beiden Auswahlfeldern am oberen Ende eine zeitliche Beschränkung der Aktivitäten angegeben werden.
Mit dem darunterlegenden Auswahlfeld, kann bestimmt werden, welche Art von Fahrtag exportiert werden soll (z.B.\ nur Nikolauszüge).
Auch kann hier ausgewählt werden, ob die Arbeitseinsätze nicht ausgegeben werden sollen.

Alle Aktivitäten, die in der Liste angezeigt werden, werden bei einem Export per Listenansicht ausgegeben.

Um eine Einzelansicht einzelner Aktivitäten auszugeben, müssen Sie die entsprechenden Aktivitäten in der Liste durch einen Klick auf das Element auswählen bzw.\ abwählen.

Um die entsprechenden Ansichten zu generieren,
wählen Sie das entsprechende Auswahlfeld.
Ebenso können Sie hier bestimmen, ob die Ausgabe als PDF-Datei auf den zuvor konfigurierten Webserver hochgeladen (\cref{einsatz:kalender:upload}),
als PDF-Datei gespeichert oder auf einem Drucker gedruckt werden soll.
Für den Export in eine PDF-Datei wird ein Fenster geöffnet, in dem Sie den Speicherort auswählen können.
Sollten Sie die Dokumente drucken wollen,
öffnet sich das Drucker-Fenster Ihres Betriebssystems,
in dem Sie weitere Einstellungen vornehmen können (Papierformat, Ausrichtung, \dots).

Die Einzelansicht kann für jede Aktivität auch im entsprechenden Fenster direkt generiert und als PDF gespeichert oder gedruckt werden.




\section{Upload-Tool}\label{einsatz:kalender:upload}
Dieses Tool gibt ihnen die Möglichkeit die Listenansicht des Einsatzplans als PDF-Datei auf einen Webserver hochzuladen.
Dies kann manuell oder automatisch bei jedem Speichern geschehen.
Zur Konfiguration befindet sich im \aktion{Datei}-Menü ein Punkt \aktion{Eigenschaften}.
Durch diesen öffnet sich der Abschnitt in \cref{fig:einsatz:kalender:upload}.
Relevant ist hier der obere Teil.
\begin{figure}[!h]
  \centering
	\includegraphics[width=0.5\textwidth]{img/eigenschaften}
	\caption{Des Fenster der Dateieigenschaften}
	\label{fig:einsatz:kalender:upload}
\end{figure}
\begin{itemize}
  \item
  Mit dem ersten Haken wird das Tool aktiviert.
  \item
  Der zweite dient dazu einzustellen, ob die Datei automatisch hochgeladen wird, wenn die lokale Datei gespeichert wird.
  Diese Funktion ist nur verfügbar, wenn sie in den Programmeinstellungen aktiviert wurde (siehe \cref{epl:allg:einstellungen}).
  \item
  Unter Server, Pfad und ID geben Sie die Daten an, die Ihnen von Ihrem Webmaster mitgeteilt wurden.
  Standardmäßig wird HTTPS als Protokoll verwendet.
  \item
  Mit dem Knopf können Sie testen, ob die Eingaben korrekt sind und eine Verbindung zum Server aufgebaut werden kann.
  Dabei werden keine Daten der Aktivitäten übermittelt.
\end{itemize}
Die drei letzten Einstellungen werden benötigt, wenn die Listenansicht automatisch hochgeladen werden soll.
Sie können einstellen, in welchem Zeitraum die Aktivitäten liegen müssen, damit Sie eingeschlossen werden.
Ebenso können Sie auswählen, ob Arbeitseinsätze auch ausgegeben werden sollen.

\begin{hinweis}
  Beim Verwenden der Export-Funktion aus \cref{einsatz:kalender:export} können Sie Beschränkungen festlegen, die unabhängig von diesen Einstellungen sind.
\end{hinweis}

\begin{neu}
\begin{hinweis}
  Bei einer unsicheren Serververbindung (d.h.\ Sie verwenden HTTP als Protokoll und nicht HTTPS),
  wird der automatische Upload aus Sicherheitsgründen deaktiviert.
  Der manuelle Upload bleibt aber weiterhin möglich.
\end{hinweis}
\end{neu}

\chapter{Fahrtage}
\section{Allgemeines und Personal}
Im oberen Bereich des Fensters können die grundlegenden Daten für einen Fahrtag eingegeben werden.
Eine Markierung als \emph{wichtig} wird im Export des Fahrtages durch ein rotes Kästchen in der Tabelle gekennzeichnet.

Die Wagenreihung wirkt sich auf die Anzeige der Reservierungen aus.
Es stehen einige häufige Kombinationen der Wagen bereits zur Verfügung, sie können aber auch von Hand geändert werden.
Dabei müssen die Wagen durch Kommata voneinander getrennt werden.
Sind die Zeiten für den Fahrtag noch unbekannt, können Sie das mit dem entsprechenden Häkchen kenntlich machen.
Dies wird dann auch entsprechend ausgegeben.

Der Anlass eines Fahrtages wird in der Kalenderansicht angezeigt.
Sodass bei einem Sonderzug schnell der Grund erkennbar ist.

In den vier Listen können Tf, Zf, Zub und Begl.o.b.A.\ und Service-Personal eingetragen werden.
Es können nur Personen eingetragen werden, die die entsprechende Funktion haben (Tf, Tb und Zf).
Eine Ausnahme hiervon ist unter anderem für Ausbildungszwecke möglich.
Dazu muss einer der folgenden Begriffe in der Bemerkung gefunden werden:
\begin{itemize}
	\item Azubi
	\item Ausbildung
	\item Tf-Ausbildung
	\item Zf-Ausbildung
	\item Tf-Unterricht
	\item Zf-Unterricht
	\item Weiterbildung
\end{itemize}
Ein Zugbegleiter ohne Ausbildung wird automatisch als Begl.o.b.A.\ geführt, sodass hier keine Unterscheidung vorgenommen werden muss.
Die Namen müssen wie folgt eingeben werden: \texttt{Name; Bemerkung}.

Es können nur Personen eingetragen werden, die im System bereits registriert wurden
(näheres im Kapitel über die Personalverwaltung).
Es sind aber Ausnahmen möglich, wenn einer der folgenden Begriffe in der Bemerkung vorkommt:
\begin{itemize}
	\item Extern
	\item Führerstand
	\item FS
	\item Schnupperkurs
	\item ELF
	\item Ehrenlokführer
	\item ELF-Kurs
\end{itemize}
Darüber hinaus können weitere Bemerkungen eingegeben werden, die auch durch Strichpunkte voneinander getrennt sein dürfen.


\section{Reservierungen}
Im unteren Teil besteht die Möglichkeit die Reservierungen zu verwalten.

Es wird angezeigt, zu welchen Grad die einzelnen Klassen jeweils belegt sind. Hier wird mit den Reservierungen gerechnet, denen ein fester Sitzplatz zugeordnet wurde.

Wenn Sie zu einer Reservierung noch keinen Sitzplatz angegeben haben, wird diese Reservierung nicht zur Berechnung der Belegung herangezogen.
Allerdings wird die Reservierung in die Gesamtzahl mit einbezogen.
Es kann also folgendes vorkommen:
\begin{verbatim}
	Belegung 1. Klasse  10/40 (25\%)
	Belegung 2. Klasse          --
	Belegung 3. Klasse  30/60 (50\%)
	Belegung Gesamt    60/100 (60\%)
\end{verbatim}

Eine Reservierung können Sie bearbeiten, indem Sie doppelt auf den Eintrag in der Liste klicken.
Sie wird dann in das Formular geladen. Dort können Sie die zugehörigen Information ändern.
Die Sitzplätze müssen in folgendem Format angegeben werden: Zuerst der Wagen und dann die Sitzplätze durch Kommata getrennt.
Aufeinanderfolgende Sitzplatznummern können Sie durch das Format Von-Bis schreiben (Beispiel: \texttt{208: 60-12}).
Wenn Sie Plätze in mehreren Wagen angeben möchten, müssen Sie die Plätze durch einen Strickpunkt voneinander trennen
(Beispiel: \texttt{204: 1-40; 217: 1-30}).

Für die Reservierung können bis zu zwei Teilstrecken angegeben werden, für welche die Reservierung dann gilt.

\paragraph{Automatische Sitzplatzverteilung}
Wenn Sie bei der Art des Fahrtags eine Nikolausfahrt ausgewählt haben, haben Sie die Möglichkeit die Sitzplätze automatisch durch einen Druck auf den Knopf ``Sitzplätze verteilen'' zu verteilen.
Der Vorgang dauert in der Regel wenige Sekunden.
Es kann aber auch unter Umständen mehrere Minuten dauern.
Warten sie diese Zeit bitte ab!
Wir arbeiten ständig an einer Verbesserung des Algorithmus zur Verteilung der Sitzplätze.

HINWEIS: Aktivieren Sie das Feld ``Alle Kombinationen berechnen'' nur, wenn Sie wenige Reservierungen angegeben haben (maximal 5-7),
denn durch diese Auswahl wird der Rechenaufwand erheblich vergrößert!
Eine grafische Ansicht der Sitzplatzverteilung ist im Moment leider noch nicht möglich.

\paragraph{Reservierungen exportieren}
Da bei Nikolausfahrten sehr viele Reservierungen vorliegen, werden Sie nicht bei der Einzelansicht von Fahrtagen angegeben.
Stattdessen können Sie, wie bei allen anderen Fahrtagen auch,
die Funktion ``Reservierungen drucken \dots'' und ``Reservierungen als PDF sichern \dots'' im Menü Fahrtag nutzen.
Bei dieser Funktion werden dann die Reservierungen nach Wagen geordnet ausgegeben.
Auch die Information über einen anderen als den Standard-Zustieg wird ausgegeben.


\section{Weiteres Personal}
In der Tabelle erhalten Sie eine Übersicht über alle Personen, die für diesen Fahrtag eingetragen wurden.
Zu jeder Person wird die Aufgabe angezeigt, die sie durchführt.
Die Arbeitszeit wird dann auf das entsprechende Konto angerechnet.
Es stehen die standardmäßigen Aufgaben zum Einstellen zur Verfügung.

Personen die betriebliche und regelmäßige Aufgaben haben und somit bereits im Reiter Personal in eine Liste eingetragen wurden, werden hier auch angezeigt.
Allerdings können nur die Einsatzzeiten dieser Personen verändert werden und auch nicht gelöscht werden.
Diese Personen müssen in den entsprechenden Listen gelöscht werden.


Die Spalten ``Beginn'' und ``Ende'' dienen dazu, dass man Uhrzeiten eingeben kann, wenn Personen nicht die angegebene Zeit geholfen haben.
So kann man eine frühere Endzeit eingeben, wenn zum Beispiel eine Person nur am Vormittag half.
Die Zeit für die Lokführer wird automatisch anhand des Dienstbeginns für Tf berechnet, sodass hier kein manueller Eintrag vorgenommen werden muss.

\chapter{Arbeitseinsätze}
Die Eingabe der Daten erfolgt hier analog zur Eingabe der Daten in das Fenster für einen Fahrtag.
Nur dass hier alle Namen in die Liste eingetragen werden und auch aus ihr entfernt werden können.

\chapter{Personalplaner}\label{personal:mitglieder}
Im Kopf des Fensters aus \cref{fig:personal:mitglieder} kann eingestellt werden welche Personen angezeigt werden sollen.
Ebenso kann durch einen Knopfdruck eine E-Mail an alle angezeigten Personen erstellt werden.
Werden bei dieser Aktion Personen gefunden, für die keine Mailadresse angegeben ist,
so können die hinterlegten Postadressen in einer CSV-Datei gespeichert werden.
Diese Daten können dann z.B.\ für Serienbriefe genutzt werden.


\begin{figure}[!h]
	\includegraphics[width=\textwidth]{img/personal-liste}
	\caption{Die Mitgliederliste mit allen ausgewählten Mitgliedern.}
	\label{fig:personal:mitglieder}
\end{figure}

\begin{neu}
Die Einzelansicht einer Person kann über einen Doppelklick auf eine Zelle der entsprechenden Zeile geöffnet werden.

Über eine ent- bzw.\ zusammenfaltbare Liste auf der linken Seite des Fensters kann eingestellt werden,
welche Daten in der Tabelle angezeigt werden.
\end{neu}
Im unteren Bereich des Fensters wird eine kurze Statistik
der aktuellen und ausgetretenen Mitglieder angezeigt.

\paragraph{Export}
Die Tabelle kann über die Knöpfe \button{Tabelle drucken} und \button{Tabelle als PDF speichern} ausgegeben werden.
Über das Menü \aktion{Exportieren} steht noch ein Export als CSV-Datei zur Verfügung (hier werden alle gespeicherten Daten exportiert),
sodass die Daten in anderen Programmen verarbeitet werden können.
Ebenso besteht die Möglichkeit über den Eintrag \aktion{Stammdatenblätter} die Daten eines Mitglieds bzw.\ aller angezeigten Personen auszugeben.
Hierbei wird für jede Person eine Seite erzeugt, die in einem gemeinsamen Dokument gespeichert werden.

\part{Export}
\chapter{Allgemeines zum Export}
\section{Allgemein}
Mit den beiden Auswahlfeldern am oberen Ende kann eine zeitliche Beschränkung der Aktivitäten angegeben werden.

Mit dem darunterlegenden Auswahlfeld, kann bestimmt werden, welche Art von Fahrtag exportiert werden soll (z.B.\ nur Nikolausszüge).
Auch kann hier ausgewählt werden, ob die Arbeitseinsätze nicht ausgegeben werden sollen.

Alle Aktivitäten, die in der Liste angezeigt werden, werden bei einem Export per Listenansicht ausgegeben.
Um eine Einzelansicht einzelner Aktivitäten auszugeben, müssen Sie diese Aktivität bzw.\ Aktivitäten in der Liste durch einen Klick auf das Element auswählen oder abwählen.

Um die entsprechenden Ansichten zu generieren, müssen Sie die Haken an den entsprechenden Auswahlkästen setzen.
Ebenso können Sie hier bestimmen, ob die Ausgabe als PDF-Datei auf den zuvor konfigurierten Webserver hochgeladen wird, als PDF-Datei gespeichert wird oder auf einem Drucker gedruckt wird.
Für den Export in eine PDF-Datei wird ein Fenster geöffnet, in dem Sie den Speicherort auswählen können.
Sollten Sie die Dokumente drucken wollen, öffnet sich das Drucker-Fenster Ihres Betriebssystems, in dem Sie weitere Einstellungen vornehmen können (Papierformat, Ausrichtung, \dots).


\section{Personal}
Um Daten aus der Personalübersicht zu exportieren, öffnen Sie die Personalübersicht und klicken Sie auf die Knöpfe ``Als PDF speichern\dots'' oder ``Drucken\dots''.
Alle angezeigten Spalten werden dann exportiert.
Weitere Informationen gibt es im Kapitel über das Personalfenster.


\section{Mitglieder}
(fehlt)


\section{Reservierungen}
Ein Fahrtag bietet noch eine weitere Methode Informationen zu exportieren.
Diese Funktion wird vor allem bei Nikolausfahrten benötigt.
Denn bei dieser Funktion werden viele Reservierungen übersichtlich nach Wagen sortiert ausgegeben.
Ebenso werden in der Einzelansicht keine Reservierungen bei Nikolausfahrten angegeben.

Diese Funktion ist über das Menü ``Fahrtag'' erreichbar.
Es gibt die Möglichkeit das Dokument entweder als PDF zu speichern oder auf einem Drucker zu drucken.

Das exportierte Dokument enthält folgende Informationen zu einer Reservierung: Name, Anzahl Sitzplätze, Sitzplätze, Kommentare/Bemerkungen sowie den Zustieg, sofern er verschieden von Ottweiler ist.

\chapter{Datei-Eigenschaften}\label{einsatz:eigen}
\section{Upload-Tool}\label{einsatz:eigen:upload}
Dieses Tool gibt ihnen die Möglichkeit die Listenansicht des Einsatzplans als PDF-Datei auf einen Webserver hochzuladen.
Dies kann manuell oder automatisch bei jedem Speichern geschehen.
Zur Konfiguration befindet sich im Datei-Menü ein Punkt \aktion{Eigenschaften}.
Durch diesen öffnet sich der Abschnitt in \cref{fig:uploadtool}.
\begin{figure}[!h]
  \centering
	\includegraphics[width=0.5\textwidth]{img/eigenschaften_upload}
	\caption{Des Fenster der Dateieigenschaften}
	\label{fig:uploadtool}
\end{figure}
\begin{itemize}
  \item
  Mit dem ersten Haken wird das Tool aktiviert.
  \item
  Der zweite dient dazu einzustellen, ob die Datei automatisch hochgeladen wird, wenn die lokale Datei gespeichert wird.
  Diese Funktion ist nur verfügbar, wenn sie in den Programmeinstellungen aktiviert wurde (siehe \cref{einstellungen}.
  \item
  Unter Server, Pfad und ID geben Sie die Daten an, die Ihnen von Ihrem Webmaster mitgeteilt wurden.
  Bei Server ist das Protokoll mit anzugeben, sofern es sich nicht um \texttt{https} handelt, also \texttt{http} verwendet wird.
  \item
  Mit dem Knopf können Sie testen, ob die Eingaben korrekt sind und eine Verbindung zum Server aufgebaut werden kann.
  Dabei werden keine Daten der Aktivitäten übermittelt.
\end{itemize}
Die drei letzten Einstellungen werden benötigt, wenn die Listenansicht automatisch hochgeladen werden soll.
Sie können einstellen, in welchem Zeitraum die Aktivitäten liegen müssen, damit Sie eingeschlossen werden.
Ebenso können Sie auswählen, ob Arbeitseinsätze auch ausgegeben werden sollen.

\begin{hinweis}
  Beim Verwenden der Export-Funktion aus \cref{export} können Sie Beschränkungen festlegen, die unabhängig von diesen Einstellungen sind.
\end{hinweis}

\part{Personalplaner}

\part{Verschiedenes}
\chapter{Einstellungen}\label{epl:alg:einstellungen}
Im Hauptmenü der Programme haben Sie unter dem Eintrag \aktion{Einstellungen \dots}
die Möglichkeit Einstellungen für das Programm festzulegen
(siehe \cref{fig:einstellungen}).
\begin{figure}[!h]
	\includegraphics[width=\textwidth]{img/einstellungen}
	\caption{Das Fenster für die Programmeinstellungen.}
	\label{fig:einstellungen}
\end{figure}



\section{Updates}
Sie können eine automatische Suche nach neuen Versionen des Programms einstellen.
Ebenso können Sie sich die Informationen zur aktuellen Programmversion ansehen.

Die Funktion ist standardmäßig aktiviert.



\section{Auto-Sicherung}
Hier haben Sie die Möglichkeit einzustellen, ob das Program eine automatische Sicherung durchführen soll.
Bei dieser Funktion wird ihre originale Datei nicht überschrieben.
Stattdessen wird eine Datei mit dem gleichen Namen erstellt, der um \datei{.autosave.ako} ergänzt ist.
Diese Datei wird automatische beim manuellen Speichern entfernt.
Sollte das Programm unerwartet abstürzen, können Sie diese Datei direkt öffnen.

Standardmäßig ist diese Funktion deaktiviert!



\section{Online-Tool}
Hier können Sie einstellen, ob beim manuellen Speichern eine Listenansicht auf den Server hochgeladen wird.
Dies setzt voraus, dass die entsprechende Option in den Einstellungen für die \EPL-Datei auch aktiviert ist
(siehe \cref{einsatz:kalender:upload}).
Unabhängig von diesen Einstellungen können Sie die Ansicht jederzeit über die Exportfunktion (\cref{einsatz:kalender:export}) hochladen.

\begin{hinweis}
  Diese Funktion wird nur im Programm \Einsatz verwendet.
  Wenn die die Datei in \Personal speichern, wird keine Datei auf den Server geladen!
\end{hinweis}


\section{Anzeige Namen}
Es kann eingestellt werden, wie die Namen der Personen in der Mitgliederübersicht angezeigt und sortiert werden.
Die Änderung wird beim nächsten Aktualisieren der Mitgliederverwaltung wirksam und erfordert keinen Neustart.

\chapter{Informationen zu Aktualisierungen}\label{epl:update}
\section{Update von Version 1.6 auf 1.7}
\label{epl:update:1.6-1.7}
Mit Version 1.7.0 wurde das frühere Programm \Einsatz in die beiden Programme \Einsatz und \Personal geteilt.
Das neue Programm \Personal dient ausschließlich der Verwaltung der Vereinsmitglieder.
Die Mitgliederdaten können im Programm \Einsatz nur noch begrenzt geändert werden, um die Übersichtlichkeit zu verbessern.

Ebenso wurde ein optionaler Passwortschutz eingeführt.
Siehe dazu \cref{epl:allg:datei:passwort}.

Darüber hinaus können verschiedene weitere Mitgliedsdaten gespeichert werden und Aktivitäten abgesagt werden.

Personen können ab sofort auch mehrfach bei einer Aktivität eingetragen werden,
unabhängig von der Kategorie.

\section{Update von Version 1.5 auf 1.6}
\label{epl:update:1.5-1.6}
In Version 1.6 wurden verschiedene Neuerungen eingeführt, die eine gesonderte Betrachtung erfordern.
Diese Neuerungen werden im Folgenden beschrieben.


\paragraph{Personaldaten}
In der Personalverwaltung (siehe \cref{personal:person}) gibt es neben den verschiedenen persönlichen Daten auch ein Feld für die \emph{Betriebsdiensttauglichkeit}.
Dieses ist dafür vorgesehen sicherzustellen, dass nur Personal eingesetzt wird, dessen medizinische Tauglichkeit noch gültig ist.
Es ist nicht möglich Personal mit abgelaufener Tauglichkeit für eine betriebliche Aufgabe einzutragen.
Ist der Zeitpunkt für die Untersuchung unbekannt, wird das Personal aus medizinischer Sicht als tauglich angesehen.
Es liegt dann vollständig beim Bediener dies zu überwachen.

Ebenso kann für eine Person ein \emph{Austrittsdatum} angegeben werden.
Während dies im Allgemeinen nicht benötigt wird, kann diese Funktion dennoch bei (angehenden) ehemaligen Mitgliedern verwendet werden.
Somit kann die Person weiterhin im System registriert sein und muss nicht aus allen Aktivitäten entfernt werden.
Für diese Personen werden nach dem Austritt selbstverständlich keine Mindeststunden mehr berechnet.
Ebenso können sie nach dem Austritt auch nicht mehr für Arbeitseinsätze oder Fahrtage eingetragen werden.

Eine weitere Neuerung ist die \emph{Mitgliedsnummer}.
Beim erstmaligen Öffnen einer Datei, die mit einer früheren Programmversion erstellt wurde, wird jeder registrierten Person eine fortlaufende Nummer zugewiesen.
Diese Nummer kann nachträglich geändert werden, um sie an die eigene Bedürfnisse anzupassen.
Da naturgemäß jede Nummer nur einmal vergeben werden kann,
kann es beim Ändern der Nummer zu Kollisionen kommen.
Deshalb sollte bei der größten zuzuweisenden Nummer mit dem Ändern begonnen werden.
Dadurch wird die Nummer wieder für andere Personen freigegeben.
Wird eine Person neu angelegt,
so wird ihr automatisch die nächste Nummer größer als die aktuell größte vergebene Nummer zugewiesen.


\paragraph{Personaleintrag bei Aktivitäten}
Bei früheren Programmversionen war es nicht möglich eine Person mehrfach in einer Aktivität (Fahrtag oder Arbeitseinsatz) einzutragen.
Dies ist ab sofort möglich.
Allerdings muss hierzu die entsprechende Aufgabe der Person verschieden sein.
So kann zum Beispiel eine Person als Tf und als Zf eingetragen werden.
Allerdings sollte dann die Zeiten für die entsprechenden Tätigkeiten angepasst werden,
da sonst die Zeiten der Person mehrfach angerechnet werden (jeweils für jede Tätigkeit).


\appendix
\part{Anhang}
\chapter{Begrifflichkeiten}
\section{Abkürzungen}\label{abkuerzungen}
\begin{tabularx}{\textwidth}{l|X}
  Abkürzung & Langform \\
  \hline
  \hline
  Tf & Triebfahrzeugführer (auch Lokführer genannt)\\
  \hline
  Tb & Triebfahrzeugbegleiter \\
  \hline
  Zf & Zugführer (teilweise auch Zugchef genannt)\\
  \hline
  Zub	& Zugbegleiter \\
  \hline
  Zs & Zugschaffner \\
  \hline
  Begl.o.b.A. & Begleiter ohne betriebliche Ausbildung \\
  \hline
  ELF &	Ehrenlokführerschnupperkurs \\
\end{tabularx}


\section{Begriffsdefinitionen}\label{glossar}
\begin{tabularx}{\textwidth}{l|X}
  Begriff	& Definition \\
  \hline
  \hline
  Aktivität &
    Ein Fahrtag oder ein Arbeitseinsatz. \\
  \hline
  Fahrtag	&
    Eine Aktivität, bei der vor allem der Zugbetrieb eine Rolle spielt.
    So gibt es hier Möglichkeiten Personal für spezielle Aufgaben anzugeben, die bei Arbeitseinsätzen nicht gegeben sind.\\
  \hline
  Arbeitseinsatz &
    Eine Aktivität, bei der eine Arbeit im Mittelpunkt steht, bei der es vornehmlich nicht um den Zugbetrieb geht.
    Zum Beispiel vorbereiten des Museumszuges oder Vegetationsarbeiten. \\
  \hline
  ELF &
    Ein Kurs, bei dem ein bis zwei Personen einen Einblick in die Welt des Tf bekommt.
    Die Personen nehmen an einem theoretischen Unterricht teil und am zweiten Tag an der praktischen Ausbildung mit Museumszugbetrieb.\\
  \hline
  Listenansicht &
    Ein Dokument, bei dem eine Übersicht über viele Aktivitäten geben wird.
    Hier finden sich alle Personen, die der Aktivität zugeteilt wurden.
    Informationen zu Reservierungen werden nur sehr begrenzt gegeben. \\
  \hline
  Einzelansicht &
  	Eine ausführliche Information zu einer einzelnen Aktivität.
    Hier werden alle Reservierung ausführlich angegeben. \\
  \hline
  Kategorie/Aufgabe &
  	Beschreibt in einem kurzen Stichwort, welche Aufgabe die Person verrichtet hat und auf welches Stundenkonto die Stunden angerechnet werden. \newline
    Es gibt: Tf, Tb, Zf, Zugbegleiter (Zub), Service, Werkstatt, Zug Vorbereiten, Büro, Ausbildung und Sonstiges
\end{tabularx}


%\backmatter
\chapter{Versionshistorie}\label{version}
\section{Version 1.0}\label{versionshistorie:1:0}
\subsection{Version 1.0.0}
\label{version:1:0:0}
Veröffentlicht am 6.10.2016
\subsubsection{Allgemeines}
\begin{itemize}
  \item
  Verwaltung von Arbeitseinsätzen und Fahrtagen
  \item
  Neues Aussehen für das Startfenster: Ein Kalender ermöglicht die schnelle und sichere Navigation zu den Aktivitäten
  \item
  Verwaltung von Personal mit dessen Ausbildung
  \item
  Automatische Unterscheidung zwischen Zub. und Beil.o.b.A. aufgrund der Ausbildung der Personen
  \item
  Die Tabellen sind kompakter gestaltet und somit ist mehr Platz für Inhalt
  \item
  Unter macOS: Unterstützung der "`öffnen mit"' Funktio
  \item
  Automatisches Suchen nach Updates im Internet
  \item
  Speichert die letzte Position des Hauptfensters
  \item
  Neues Icon und Logo, das auch auf schwarzem und weißem Hintergrund gut zu erkennen ist
  \item
  Behebung von Fehlern und Problemen
\end{itemize}

\subsubsection{Fahrtag}
\begin{itemize}
  \item
  Neue Export-Funktion: Hier werden jetzt nur Reservierungen ausgegeben, nach Wagen und Name sortiert.
  \item
  Es ist möglich zusätzliches personal einzutragen, sodass die Personen nicht in die vier Gruppen eingeteilt werden müssen
  \item
  Automatische Verteilung der Sitzplätze bei einer Nikolausfahrt
  \item
  Anzeige, zu wieviel Prozent die jeweiligen Klassen und der gesamte Zug belegt sind
\end{itemize}

\subsubsection{Personal}
\begin{itemize}
  \item
  Anzeige, welche person bei welchen Aktivitäten mitgeholfen hat
  \item
  Bestimmen der Einsatzzeiten des Personals für einzelne Arbeitsbereiche (Z.B. Werkstatt, Zugbegleitung)
\end{itemize}


\subsection{Version 1.0.1}
\label{version:1:0:1}
Veröffentlicht am 7.10.2016
\subsubsection{Fehlerbehebungen}
\begin{itemize}
  \item Problem beim öffnen von Dateien (Externe Personen wurden nicht geladen, Personen wurden nicht im Arbeitseinsatzfenster angezeigt)
  \item Problem bei Anzeige von Personen (Die Informationen zu den Aktivitäten wurden unter Umständen falsch dargestellt)
\end{itemize}


\subsection{Version 1.0.2}
\label{version:1:0:2}
Veröffentlicht am 13.11.2016
\begin{itemize}
  \item
  Die Wagenreihung kann jetzt auch mit Leerräumen eingegeben werden.
  \item
  Personen können jetzt auch mit "`Nachname, Vorname"' in die Listen eingegeben werden.
  \item
  Die Einträge werden jetzt in der Seitenleiste richtig sortiert. Auch werden die Daten in der richtigen Reihenfolge exportiert.
  \item
  Die Personalübersicht wurde flexibler gestaltet, auch können verschiedenen Spalten angezeigt und exportiert werden.
  \item
  Der Text wird jetzt bei Bedarf im Kalender umgebrochen.
  \item
  Wenn die Maus über dem Eintrag in der Seitenleiste verweilt, wird der Anlass der Veranstaltung angezeigt.
  \item
  Wichtige Fahrtage werden als solche in der Seitenleiste angezeigt.
  \item
  Reservierungen werden jetzt auf Plausibilität geprüft (z.B. kein Einstig am letzten Haltepunkt eines Zuges).
  \item
  Weitere Verbesserungen bei der Stabilität und Fehlerbehebungen.
\end{itemize}


\subsection{Version 1.0.3}
\label{version:1:0:3}
Veröffentlicht am 2.12.2016
\subsubsection{Verbesserungen}
\begin{itemize}
  \item
  Beim Export werden jetzt die Aufgaben der Personen angegeben, wenn sie unter sonstigem Personal gelistet wurden
  \item
  Arbeitseinsätze werden jetzt standardmäßig in der Listenansicht exportiert
  \item
  Bei Nikolausfahrten werden die Reservierungen nicht mehr auf dem Übersichtsblatt ausgegeben, sie müssen ab sofort über die Funktionen Reservierungen exportieren ausgegeben werden
\end{itemize}

\subsubsection{Fehlerbehebungen}
\begin{itemize}
  \item
  Personen können jetzt wieder in die Personallisten korrekt eingetragen und gelöscht werden, auch wenn eine Bemerkung angegeben ist
  \item
  Die Auswahl eines Datum bei der Export-Funktion für das "`Bis Datum"' funktioniert jetzt wieder wie erwartet
\end{itemize}
Weitere kleinere Verbesserungen und Fehlerbehebungen


\subsection{Version 1.0.4}
\label{version:1:0:4}
Veröffentlicht am 14.12.2016

Der Fehler, das das Programm beim Export der Personalübersicht bei manchen Benutzern abstürzt, wurde behoben.

Weitere kleine Fehlerkorrekturen und Verbesserungen.

\section{Version 1.1}\label{versionshistorie:1:1}
\subsection{Version 1.1.0}
\label{version:1:1:0}
Veröffentlicht am 21.12.2016
\subsubsection{Veränderungen}
\begin{itemize}
  \item
  Für jede Person kann eine zusätzliche Stundenanzahl für jedes Konto angegeben werden. Diese Zeiten können in der Einzelansicht einer person angegeben werden
  \item
  Es muss kein Haken mehr gesetzt werden, wenn die Reservierungen automatisch verteilt werden sollen
  \item
  Beim Export von Reservierungen wird jetzt auch der Zu- und Ausstieg angegeben
\end{itemize}

\subsubsection{Fehlerbehebungen}
\begin{itemize}
  \item
  Die Aufgabe einer Person bei einer bestimmten Aktivität wird jetzt korrekt gespeichert und geladen
  \item
  Weitere kleinere Fehlerbehebungen und Verbesserungen
\end{itemize}

\section{Version 1.2}\label{versionshistorie:1:2}
\subsection{Version 1.2.0}
\label{version:1:2:0}
Veröffentlicht am 10.10.2017
\subsubsection{Neu}
\begin{itemize}
  \item
  Die Mindeststunden können eingestellt werden, dies ist im Personalfenster möglich. Dort findet sich ein neuer Knopf mit der Bezeichnung Mindeststuden.
  \item
  Anzeige der Summe der Einsatzzeiten in der Personalübersicht und in der exportierten Tabelle. Diese Funktion ist immer aktiv und zeigt die Summe der Zeiten der jeweiligen Spalte an.
  \item
  Beim Export von Daten wird jetzt auch die Uhrzeit mit ausgegeben.
\end{itemize}

\subsubsection{Verbessert}
\begin{itemize}
  \item
  Kennzeichnung von externen Personen auf Triebfahrzeugen oder ähnlichem wurde vereinfacht und verbessert. Externe Personen können jetzt mit folgenden Begriffen gekennzeichnet werden, sodass sie als extern angesehen werden und nicht im Personalverzeichnis gesucht werden:
  \begin{itemize}
    \item „Extern“
    \item „Führerstand“
    \item „FS“
    \item „Schnupperkurs“
    \item „ELF“
    \item „Ehrenlokführer“
    \item „ELF-Kurs“
  \end{itemize}
  Hierbei ist es ausreichend, wenn einer dieser Begriffe in der Bemerkung vorkommt.
  \item
  Für registriertes Personal wurde eine ähnliche Funktion eingeführt, die verhindert, dass überprüft wird, ob eine Person die Qualifikation hat. Dies kann dann insbesondere für Ausbildungszwecke verwendet werden. Hier die Schlüsselbegriffe:
  \begin{itemize}
    \item   „Azubi“
    \item  „Ausbildung“
    \item  „Tf-Ausbildung“,
    \item  „Zf-Ausbildung“
    \item  „Tf-Unterricht“
    \item  „Zf-Unterricht“
    \item  „Weiterbildung“
  \end{itemize}
  \item
  In der Personalübersicht eines Arbeitseinsatzes und Fahrtages können jetzt die Aufgaben aus der voreingestellten Liste ausgewählt werden. Ebenso wurde ein extra Feld für die Bemerkung eingeführt. Die Zeiten können jetzt besser eingestellt werden.
  \item
  Die Sitzplätze bei Reservierungen können jetzt besser eingegeben werden. Hier wurde ein Fehler behoben.
\end{itemize}

\subsubsection{Fehlerbehebungen}
\begin{itemize}
  \item
  Beim öffnen einer Datei, wird jetzt wieder zu dem Monat gesprungen, an dem die Datei gespeichert wurde.
  \item
  Beim Öffnen eines Fahrtags wird die Bemerkung wieder geladen.
  \item
  Rechtschreibfehler korrigiert
  \item
  Weitere kleine Fehlerbehebungen und Verbesserungen
\end{itemize}

\section{Version 1.3}\label{versionshistorie:1:3}
\subsection{Version 1.3.0}
\label{version:1:3:0}
Veröffentlicht am 23.12.2017
\subsubsection{Neu}
\begin{itemize}
  \item
  Es kann jetzt eine Übersicht über die einzelnen Aktivitäten einer person ausgegeben werden, die Funktion dazu findet sich in der Einzelansicht des Personalfensters.
  \item
  Als weitere Kategorie für eine Aufgabe, kann jetzt noch "`Ausbildung"' angegeben werden.
  \item
  Es können nun auch zusätzliche Kilometer eingegeben werden, diese werden auf die automatisch berechneten Kilometer angerechnet.
\end{itemize}

\subsubsection{Verbessert}
Bei einer Reservierung kann jetzt eine zweite Teilstrecke eingegeben werden.

\subsubsection{Fehlerbehebungen}
\begin{itemize}
  \item
  Bei einer Änderung an einer Reservierung wird vor dem Schließen wieder nachgefragt, ob die Änderungen übernommen werden sollen.
  \item
  Optimierung beim Personalfenster, sodass es kleiner als bisher gemacht werden kann.
\end{itemize}


\subsection{Version 1.3.1}
\label{version:1:3:1}
Veröffentlicht am 14.03.2018
\subsubsection{Neu}
Das Arbeitseinsatzfenster wurde intelligenter. Es wählt eine bestimmte Kategorie für die Personaltabelle aus, wenn es dies aus dem Anlass folgern kann.

\subsubsection{Verbessert}
\begin{itemize}
  \item
  Wird personal für einen Fahrtag oder Arbeitseinsatz benötigt, wird dies farblich in der Einzelansicht kenntlich gemacht.
  \item
  Das Personal, dass bei einem Schnupperkurs eingetragen wurde, wird jetzt in der Listen- und Einzelansicht ausgegeben
  \item
  Externe Personen können nun auch mit "`Nachname, Vorname"' in Listen eingetragen werden.
\end{itemize}

\subsubsection{Fehlerbehebungen}
\begin{itemize}
  \item
  Probleme beim Drucken der Einzel- und Listenansicht zumindest unter macOS, wenn im Druckdialog "`als PDF speichern"' oder ähnliches gewählt wurde, sind behoben.
  \item
  Veränderungen in der Personaltabelle einer Aktivität wurde nicht korrekt übernommen.
  \item
  Fehler behoben, der es ermögliche nicht-Betriebsdienstpersonal als Zub einzutragen.
  \item
  Weitere Fehlerbehebungen, u.A. beim Darstellen der Daten nach dem Öffnen der Datei.
\end{itemize}

\section{Version 1.4}\label{versionshistorie:1:4}
\subsection{Version 1.4.0}
\label{version:1:4:0}
Veröffentlicht am 23.7.2018
\subsubsection{Neu}
\begin{itemize}
  \item
  Das Programm merkt sich die zuletzt verwendeten Dateien. Sie können unter „Datei > Zuletzt benutzt“ direkt geöffnet werden.
  \item
  Im Kalender können jetzt direkt Arbeitseinsätze an einem ausgewählten Tag erstellt werden.
\end{itemize}

\subsubsection{Verbessert}
Für eine Person kann jetzt auch noch eine zusätzliche Anzahl an Aktivitäten angegeben werden. Bisher war dies nur für Zeiten und Kilometer möglich.

\subsubsection{Fehlerbehebungen}
\begin{itemize}
  \item
  Die Druckausgabe wurde verbessert. Unter macOS ist nun das korrekte Erstellen einer PDF-Datei möglich.
  \item
  Mehrere Fehler beim Kalender wurden behoben.
  \item
  Weitere verschiedene Fehlerbehebungen und Verbesserungen, unter anderem bei der Verwaltung von Reservierungen.
\end{itemize}

\subsection{Version 1.4.1}
\label{version:1:4:1}
Veröffentlicht am 25.9.2018
\subsubsection{Neu}
Kalender zeigt jetzt den Anlass einer Aktivität und eines Fahrtages an, sofern es möglich ist.

\subsubsection{Verbessert}
\begin{itemize}
  \item
  Fahrtagfenster und Fenster für Aktivitäten optimiert, sodass sie den Platz besser nutzen.
  \item
  Verbesserter Export von Arbeitseinsätzen, sowohl in der Listen- als auch in der Einzelansicht.
  \item
  Fahrtage und Aktivitäten werden jetzt nicht mehr nur nach Datum sondern auch nach Beginn und Endzeit sortiert.
  \item
  Die Sitzplätze werden jetzt besser und schneller verteilt.
  \item
  Das Programm benötigt bei längerem Betrieb weniger Arbeitsspeicher.
  \item
  Code optimiert und kleinere Funktionen verbessert.
\end{itemize}

\subsubsection{Fehlerbehebungen}
\begin{itemize}
  \item
  Begleiter ohne betriebliche Aufgaben erschienen unter Umständen nicht in der Einzelansicht.
  \item
  Fahrtage und Aktivitäten können jetzt ohne Probleme gelöscht werden.
  \item
  Bei einem Arbeitseinsatz wird jetzt die Kategorie auch nach einem erneuten Öffnen intelligent bestimmt.
  \item
  Bei der Personalübersicht werden die Zeiten für die Ausbildung verlässlich angezeigt.
  \item
  Weitere Fehlerbehebungen zur Verbesserung der Stabilität
\end{itemize}

\section{Version 1.5}\label{versionshistorie:1:5}
\subsection{Version 1.5.0}
\label{version:1:5:0}
Veröffentlicht am 31.3.2019
\subsection{Neu}
\begin{itemize}
  \item
  Der Einsatzplan kann als PDF direkt aus dem Programm auf einen vorher konfigurierten Webserver hochgeladen werden. Dies geschieht entweder beim Speichern oder manuell.
  \item
  Automatisches Sichern: Das Programm speichert bei Wunsch nach einer bestimmten Zeit automatisch eine Backup-Datei.
  \item
  Noch unbekannte Einsatzzeiten können jetzt als solche ausgewiesen werden.
  \item
  Fahrtage und Arbeitseinsätze können direkt aus dem jeweiligen Fenster heraus gelöscht werden.
\end{itemize}

\subsubsection{Verbessert}
\begin{itemize}
  \item
  In der Personalübersicht werden jetzt auch die Mindeststunden der Personen angezeigt.
  \item
  In der Listenansicht von Arbeitseinsätzen wurden redundante Informationen entfernt.
  \item
  In der Listenansicht von Arbeitseinsätzen wird jetzt auch ein etwaiger Ort angegeben.
  \item
  Eine Aktivität wird erst nach einer Sicherheitsabfrage gelöscht.
\end{itemize}

\subsubsection{Fehlerbehebungen}
\begin{itemize}
  \item
  Beim Eintragen von Personal für einen Arbeitseinsatz wurde die Person unter Umständen nicht richtig übernommen.
  \item
  Kleinere Optimierungen und Verbesserungen
\end{itemize}


\subsection{Version 1.5.1}
\label{version:1:5:1}
Veröffentlicht am 27.11.2019
\subsubsection{Neu}
Die Einsatzzeiten einer einzelnen Person können ab sofort als PDF gespeichert und gedruckt werden.

\subsubsection{Verbessert}
\begin{itemize}
  \item
  Der Export der Aktivitäten als Listen- und Einzelansicht wurde optimiert.
  \item
  Der Export der Personaldaten als Listen- und Einzelansicht wurde verbessert.
\end{itemize}

\subsubsection{Fehlerbehebungen}
\begin{itemize}
  \item
  Beim Eintragen von Personal für einen Arbeitseinsatz wurde die Person unter Umständen nicht richtig übernommen.
  \item
  Externes Personal eines Fahrtages wird jetzt wieder im entsprechenden Fenster dargestellt.
  \item
  Die Personalübersicht bleibt sortiert, auch wenn sie aktualisiert wird.
  \item
  Ein Problem beim automatischen Speichern wurde behoben.
  \item
  Die Einstellung des Zeitraums beim Datei-Upload wird jetzt zuverlässig verwendet.
  \item
  Kleinere Verbesserungen und Fehlerbehebungen
\end{itemize}


\subsection{Version 1.5.2}
\label{version:1:5:2}
Veröffentlicht am 22.3.2020
\subsubsection{Neu}
Durch einen Doppelklick auf eine Person in der Gesamtübersicht des Personalfensters wird die entsprechende Einzelansicht angezeigt.

\subsubsection{Verbessert}
Export der Personalübersichten verbessert, indem Stunden besser formatiert werden.

\subsubsection{Fehlerbehebungen}
\begin{itemize}
  \item
  Ein Fehler wurde behoben, bei dem das Programm beim Beenden abstürzt, wenn das automatische Speichern deaktiviert war.
  \item
  Beim Export der Personaldaten werden die Dateieinstellungen übernommen.
\end{itemize}

\section{Version 1.6}\label{version:1:6}
\subsection{Version 1.6.0}
\label{version:1:6:0}
Veröffentlicht am 02.05.2020
\subsubsection{Neu}
\begin{itemize}
  \item
  Die Personalverwaltung wurde verbessert, indem jetzt alle Vereinsmitglieder aufgenommen und verwaltet werden können.
  Ebenso können verschiedene persönliche Daten und Kontaktdaten eingegeben und auch entsprechend exportiert weden.
  \item
  Eine Person kann jetzt mehrfach bei einem Arbeitseinsatz oder Fahrtag eingetragen werden,
  vorausgesetzt die Aufgabe ist jeweils verschieden.
  \item
  Anzeige der Auslastung der einzelnen Züge anhand der eingetragenen Reservierungen.
  \item
  Die Liste der Reservierungen kann nach Zügen gefiltert werden.
  \item
  Komplett überarbeitete Dokumentation.
\end{itemize}

\subsubsection{Verbessert}
\begin{itemize}
  \item
  Es gibt jetzt eine neue Kategorie "`Infrastruktur"'.
  Diese kann z.B.\ für Streckenarbeiten und Vegetationskontrollen genutzt werden.
  \item
  Die Anzahl der benötigten Lokführer kann beliebig zwischen null und zwei festgelegt werden,
  falls mit mehr als einem Triebfahrzeug gefahren wird.
  \item
  Die zusätzlichen Stunden und Mindeststunden können jetzt minutengenau eingegeben werden.
  \item
  Die Summe der Spalten in der Tabelle der Gesamtübersicht bezieht sich immer auf die aktuell angezeigten Personen.
  \item
  Beim Export von Daten können im Druckerdialog jetzt auch Seitenformat und Ausrichtung bestimmt werden.
  \item
  Unzählige Verbesserungen und Optimierungen "`unter der Haube"', um die Geschwindigkeit und den Speicherverbrauch zu optimieren.
\end{itemize}

\subsubsection{Fehlerbehebungen}
\begin{itemize}
  \item
  Fehler bei der Akzeptanz bestimmter Fahrstrecken einer Reservierung behoben.
  \item
  Die Einträge der Aktivitäten im Kalender bleiben nicht mehr markiert.
  \item
  Verschiedene kleinere Fehler behoben.
\end{itemize}



\end{document}
