\chapter{Begriffsdefinitionen}\label{glossar}
\begin{tabularx}{\textwidth}{l|X}
  Begriff	& Definition \\
  \hline
  \hline
  Fahrtag	&
    Eine Aktivität, bei der vor allem der Zugbetrieb eine Rolle spielt.
    So gibt es hier Möglichkeiten Personal für spezielle Aufgaben anzugeben, die bei Arbeitseinsätzen nicht gegeben sind.\\
  \hline
  Arbeitseinsatz &
    Eine Aktivität, bei der eine Arbeit im Mittelpunkt steht, bei der es vornehmlich nicht um den Zugbetrieb geht.
    Zum Beispiel vorbereiten des Museumszuges oder Vegetationsarbeiten. \\
  \hline
  ELF &
    Ein Kurs, bei dem ein bis zwei Personen einen Einblick in die Welt des Tf bekommt.
    Die Personen nehmen an einem theoretischen Unterricht teil und am zweiten Tag an der praktischen Ausbildung mit Museumszugbetrieb.\\
  \hline
  Listenansicht &
    Ein Dokument, bei dem eine Übersicht über viele Aktivitäten geben wird.
    Hier finden sich alle Personen, die der Aktivität zugeteilt wurden.
    Informationen zu Reservierungen werden nur sehr begrenzt gegeben. \\
  \hline
  Einzelansicht &
  	Eine ausführliche Information zu einer einzelnen Aktivität.
    Hier werden alle Reservierung ausführlich angegeben. \\
  \hline
  Kategorie/Aufgabe &
  	Beschreibt in einem kurzen Stichwort, welche Aufgabe die Person verrichtet hat und auf welches Stundenkonto die Stunden angerechnet werden.
    Es gibt: Tf, Tb, Zf, Zub, Service, Werkstatt, Zug Vorbereiten, Büro, Ausbildung und Sonstiges
\end{tabularx}
