\chapter{Einstellungen}
Unter ``Einsatzplaner'' \nach ``Einstellungen\dots'' haben Sie die Möglichkeit Einstellungen für das Programm festzulegen.

\section{Updates}
Sie können im Menü ``Einstellungen'' eine automatische Suche nach neuen Versionen des Programms einstellen.
Dies ist standardmäßig der Fall.
Ebenso können Sie sich die Informationen zur aktuellen Programmversion ansehen.


\section{Auto-Sicherung}
Hier haben Sie die Möglichkeit einzustellen, ob das Program eine automatische Sicherung durchführen soll.
Bei dieser Funktion wird ihre originale Datei nicht überschrieben.
Stattdessen wird eine Datei mit dem gleichen Namen erstellt, der um \texttt{.autosave.ako} ergänzt ist.
Diese Datei wird automatische beim Speichern entfernt.
Sollte das Programm unerwartet abstürzen, können Sie diese Datei direkt öffnen.
Standardmäßig ist diese Funktion nicht aktiviert!



\section{Online-Tool}
Hier können Sie einstellen, ob beim automatischen Speichern von Dateien diese auf den Server hochgeladen werden.
Dies passiert allerdings nur, wenn die entsprechende Option in den Einstellungen für die Einsatzplaner-Datei auch gesetzt sind.

Unabhängig von diesen Einstellungen können Sie die Datei jederzeit über die Exportfunktion hochladen.
