\chapter{Allgemeines zum Export}
\section{Allgemein}
Mit den beiden Auswahlfeldern am oberen Ende kann eine zeitliche Beschränkung der Aktivitäten angegeben werden.

Mit dem darunterlegenden Auswahlfeld, kann bestimmt werden, welche Art von Fahrtag exportiert werden soll (z.B.\ nur Nikolausszüge).
Auch kann hier ausgewählt werden, ob die Arbeitseinsätze nicht ausgegeben werden sollen.

Alle Aktivitäten, die in der Liste angezeigt werden, werden bei einem Export per Listenansicht ausgegeben.
Um eine Einzelansicht einzelner Aktivitäten auszugeben, müssen Sie diese Aktivität bzw.\ Aktivitäten in der Liste durch einen Klick auf das Element auswählen oder abwählen.

Um die entsprechenden Ansichten zu generieren, müssen Sie die Haken an den entsprechenden Auswahlkästen setzen.
Ebenso können Sie hier bestimmen, ob die Ausgabe als PDF-Datei auf den zuvor konfigurierten Webserver hochgeladen wird, als PDF-Datei gespeichert wird oder auf einem Drucker gedruckt wird.
Für den Export in eine PDF-Datei wird ein Fenster geöffnet, in dem Sie den Speicherort auswählen können.
Sollten Sie die Dokumente drucken wollen, öffnet sich das Drucker-Fenster Ihres Betriebssystems, in dem Sie weitere Einstellungen vornehmen können (Papierformat, Ausrichtung, \dots).


\section{Personal}
Um Daten aus der Personalübersicht zu exportieren, öffnen Sie die Personalübersicht und klicken Sie auf die Knöpfe ``Als PDF speichern\dots'' oder ``Drucken\dots''.
Alle angezeigten Spalten werden dann exportiert.
Weitere Informationen gibt es im Kapitel über das Personalfenster.


\section{Mitglieder}
(fehlt)


\section{Reservierungen}
Ein Fahrtag bietet noch eine weitere Methode Informationen zu exportieren.
Diese Funktion wird vor allem bei Nikolausfahrten benötigt.
Denn bei dieser Funktion werden viele Reservierungen übersichtlich nach Wagen sortiert ausgegeben.
Ebenso werden in der Einzelansicht keine Reservierungen bei Nikolausfahrten angegeben.

Diese Funktion ist über das Menü ``Fahrtag'' erreichbar.
Es gibt die Möglichkeit das Dokument entweder als PDF zu speichern oder auf einem Drucker zu drucken.

Das exportierte Dokument enthält folgende Informationen zu einer Reservierung: Name, Anzahl Sitzplätze, Sitzplätze, Kommentare/Bemerkungen sowie den Zustieg, sofern er verschieden von Ottweiler ist.
