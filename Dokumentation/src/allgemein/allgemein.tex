\chapter{Allgemeines}

Das EPL-Programmpaket dient dazu die Verwaltungstätigkeiten einer Museumseisenbahn zu verbessern.
Dazu können mit dem Programm ``Einsatzplaner'' Fahrtage und Arbeitseinsätze verwalten werden, sowie die Einsatzzeiten der Mitglieder berechnet werden.
Mit dem Programm ``Personalplaner'' kann eine Verwaltung der Mitglieder und Speicherung der entsprechenden Daten durchgeführt werden. 

\begin{figure}[h!]
  \centering
  \begin{subfigure}{0.3\textwidth}
    \includegraphics[width=\textwidth]{../../Icon/EPL.png}
    \caption{Logo des Prgrammpakets}
  \end{subfigure}
    
  \begin{subfigure}{0.3\textwidth}
    \includegraphics[width=\textwidth]{../../Icon/Einsatzplaner.png}
    \caption{Logo des Einsatzplaners}
  \end{subfigure}
  \begin{subfigure}{0.3\textwidth}
    \includegraphics[width=\textwidth]{../../Icon/Personalplaner.png}
    \caption{Logo des Personalplaners}
  \end{subfigure}
  \caption{Die Logos des Programmpakets}
\end{figure}
%
Fahrtage unterschieden sich von Arbeitseinsätzen dadurch, dass das Personal direkt bestimmten Aufgabengebieten zugeordnet werden kann.
Ebenso können Reservierungen eingetragen werden.

\section{Die EPL-Datei}
\subsection{Passwortschutz}

\subsection{\neu{Schreibschutz}}\label{schreibschutz}
\neu{
Beim Öffnen einer Datei \texttt{X.ako} wird automatisch eine neue Datei mit dem Namen \texttt{X.ako.lock} erstellt.
Durch diese Datei wird sichergestellt, dass nicht mehrere Personen gleichzeitig an der Datei \texttt{X.ako} arbeiten.
Wird eine bereits geöffnete Datei erneut geöffnet (vom gleichen oder fremden Benutzer),
kann sie nur schreibgeschützt geöffnet werden und Änderungen können nicht gespeichert werden.
Beim Schließen einer Datei bzw.\ beim Beenden des Programms wird die Datei \texttt{X.ako.lock} automatisch wieder gelöscht.
Danach kann ein anderer Benutzer die Datei wieder uneingeschränkt öffnen.

\hinweis{Der Schreibschutz steht nur effektiv zur Verfügung, wenn alle Beteiligten Version 1.6.2 oder neuer des Einsatzplaners verwendet.
Wird eine frühere Version verwendet, kann die Datei allerdings weiterhin uneingeschränkt geöffnet und bearbeitet werden!}

\hinweis{Die oben beschriebene Datei sollte unter normalen Umständen  nicht manuell gelöscht werden, da dies zu Datenverlusten führen kann.
Löschen Sie die Datei nur, wenn Sie sicherstellen können, dass keine weiteren Benutzer auf die Datei zugreifen.}
}
