\chapter{Fahrtage}
\section{Allgemeines und Personal}
Im oberen Bereich des Fensters können die grundlegenden Daten für einen Fahrtag eingegeben werden.
Eine Markierung als \emph{wichtig} wird im Export des Fahrtages durch ein rotes Kästchen in der Tabelle gekennzeichnet.

Die Wagenreihung wirkt sich auf die Anzeige der Reservierungen aus.
Es stehen einige häufige Kombinationen der Wagen bereits zur Verfügung, sie können aber auch von Hand geändert werden.
Dabei müssen die Wagen durch Kommata voneinander getrennt werden.
Sind die Zeiten für den Fahrtag noch unbekannt, können Sie das mit dem entsprechenden Häkchen kenntlich machen.
Dies wird dann auch entsprechend ausgegeben.

Der Anlass eines Fahrtages wird in der Kalenderansicht angezeigt.
Sodass bei einem Sonderzug schnell der Grund erkennbar ist.

In den vier Listen können Tf, Zf, Zub und Begl.o.b.A.\ und Service-Personal eingetragen werden.
Es können nur Personen eingetragen werden, die die entsprechende Funktion haben (Tf, Tb und Zf).
Eine Ausnahme hiervon ist unter anderem für Ausbildungszwecke möglich.
Dazu muss einer der folgenden Begriffe in der Bemerkung gefunden werden:
\begin{itemize}
	\item Azubi
	\item Ausbildung
	\item Tf-Ausbildung
	\item Zf-Ausbildung
	\item Tf-Unterricht
	\item Zf-Unterricht
	\item Weiterbildung
\end{itemize}
Ein Zugbegleiter ohne Ausbildung wird automatisch als Begl.o.b.A.\ geführt, sodass hier keine Unterscheidung vorgenommen werden muss.
Die Namen müssen wie folgt eingeben werden: \texttt{Name; Bemerkung}.

Es können nur Personen eingetragen werden, die im System bereits registriert wurden
(näheres im Kapitel über die Personalverwaltung).
Es sind aber Ausnahmen möglich, wenn einer der folgenden Begriffe in der Bemerkung vorkommt:
\begin{itemize}
	\item Extern
	\item Führerstand
	\item FS
	\item Schnupperkurs
	\item ELF
	\item Ehrenlokführer
	\item ELF-Kurs
\end{itemize}
Darüber hinaus können weitere Bemerkungen eingegeben werden, die auch durch Strichpunkte voneinander getrennt sein dürfen.


\section{Reservierungen}
Im unteren Teil besteht die Möglichkeit die Reservierungen zu verwalten.

Es wird angezeigt, zu welchen Grad die einzelnen Klassen jeweils belegt sind. Hier wird mit den Reservierungen gerechnet, denen ein fester Sitzplatz zugeordnet wurde.

Wenn Sie zu einer Reservierung noch keinen Sitzplatz angegeben haben, wird diese Reservierung nicht zur Berechnung der Belegung herangezogen.
Allerdings wird die Reservierung in die Gesamtzahl mit einbezogen.
Es kann also folgendes vorkommen:
\begin{verbatim}
	Belegung 1. Klasse  10/40 (25\%)
	Belegung 2. Klasse          --
	Belegung 3. Klasse  30/60 (50\%)
	Belegung Gesamt    60/100 (60\%)
\end{verbatim}

Eine Reservierung können Sie bearbeiten, indem Sie doppelt auf den Eintrag in der Liste klicken.
Sie wird dann in das Formular geladen. Dort können Sie die zugehörigen Information ändern.
Die Sitzplätze müssen in folgendem Format angegeben werden: Zuerst der Wagen und dann die Sitzplätze durch Kommata getrennt.
Aufeinanderfolgende Sitzplatznummern können Sie durch das Format Von-Bis schreiben (Beispiel: \texttt{208: 60-12}).
Wenn Sie Plätze in mehreren Wagen angeben möchten, müssen Sie die Plätze durch einen Strickpunkt voneinander trennen
(Beispiel: \texttt{204: 1-40; 217: 1-30}).

Für die Reservierung können bis zu zwei Teilstrecken angegeben werden, für welche die Reservierung dann gilt.

\paragraph{Automatische Sitzplatzverteilung}
Wenn Sie bei der Art des Fahrtags eine Nikolausfahrt ausgewählt haben, haben Sie die Möglichkeit die Sitzplätze automatisch durch einen Druck auf den Knopf ``Sitzplätze verteilen'' zu verteilen.
Der Vorgang dauert in der Regel wenige Sekunden.
Es kann aber auch unter Umständen mehrere Minuten dauern.
Warten sie diese Zeit bitte ab!
Wir arbeiten ständig an einer Verbesserung des Algorithmus zur Verteilung der Sitzplätze.

HINWEIS: Aktivieren Sie das Feld ``Alle Kombinationen berechnen'' nur, wenn Sie wenige Reservierungen angegeben haben (maximal 5-7),
denn durch diese Auswahl wird der Rechenaufwand erheblich vergrößert!
Eine grafische Ansicht der Sitzplatzverteilung ist im Moment leider noch nicht möglich.

\paragraph{Reservierungen exportieren}
Da bei Nikolausfahrten sehr viele Reservierungen vorliegen, werden Sie nicht bei der Einzelansicht von Fahrtagen angegeben.
Stattdessen können Sie, wie bei allen anderen Fahrtagen auch,
die Funktion ``Reservierungen drucken \dots'' und ``Reservierungen als PDF sichern \dots'' im Menü Fahrtag nutzen.
Bei dieser Funktion werden dann die Reservierungen nach Wagen geordnet ausgegeben.
Auch die Information über einen anderen als den Standard-Zustieg wird ausgegeben.


\section{Weiteres Personal}
In der Tabelle erhalten Sie eine Übersicht über alle Personen, die für diesen Fahrtag eingetragen wurden.
Zu jeder Person wird die Aufgabe angezeigt, die sie durchführt.
Die Arbeitszeit wird dann auf das entsprechende Konto angerechnet.
Es stehen die standardmäßigen Aufgaben zum Einstellen zur Verfügung.

Personen die betriebliche und regelmäßige Aufgaben haben und somit bereits im Reiter Personal in eine Liste eingetragen wurden, werden hier auch angezeigt.
Allerdings können nur die Einsatzzeiten dieser Personen verändert werden und auch nicht gelöscht werden.
Diese Personen müssen in den entsprechenden Listen gelöscht werden.


Die Spalten ``Beginn'' und ``Ende'' dienen dazu, dass man Uhrzeiten eingeben kann, wenn Personen nicht die angegebene Zeit geholfen haben.
So kann man eine frühere Endzeit eingeben, wenn zum Beispiel eine Person nur am Vormittag half.
Die Zeit für die Lokführer wird automatisch anhand des Dienstbeginns für Tf berechnet, sodass hier kein manueller Eintrag vorgenommen werden muss.
