\chapter{Übersichtskalender}
\section{Anlegen von Aktivitäten}
Um einen neuen Fahrtag anzulegen, klicken Sie auf den Knopf ``Neuer Fahrtag''.
Es wird ein Fahrtag mit dem aktuellen Datum angelegt.
Er wird in der Seitenleiste angezeigt.
Die Fahrtage werden, um einen schnelleren Überblick zubekommen in verschiedenen Farben anhand ihrer Art eingefärbt.

Ein neuer Arbeitseinsatz kann mit Hilfe des Knopfes ``Neuer Arbeitseinsatz'' angelegt werden, er wird ebenso wie die Fahrtage in der Liste und im Kalender angezeigt.

Ein Arbeitseinsatz kann auch direkt mit dem entsprechenden Datum durch einen Klick auf ``+'' im Kalender erstellt werden.


\section{Löschen von Aktivitäten}
Um eine Aktivität zu löschen, müssen Sie diese in der Seitenleiste auswählen und dann auf den Knopf ``Ausgewählten Eintrag löschen'' klicken.



\section{Navigation im Kalender}
Zurück: Blättert im Kalender einen Monat zurück\\
Vor:		Blättert im Kalender einen Monat vor \\
Heute:	Springt zum aktuellen Tag im Kalender \\
Eingabefeld:	Hier kann ein Datum direkt eingegeben werden


\section{Exportieren von Objekten}
Um Objekte zu exportieren, klicken Sie entweder auf den Knopf „Exportieren“ oder wählen im Menü ``Datei'' \nach ``Drucken\dots'' aus.
Alternativ ist diese Funktion auch mit dem Tastaturbefehl \texttt{cmd+P} beziehungsweise \texttt{ctrl+P} zu erreichen.
Weiteres hierzu finden Sie im Kapitel über die Export-Funktion.


\section{Das Datei-Menü}
Im folgenden werden die Einträge des ``Datei'' Menüs beschrieben.
\begin{description}
  \item[Neu]
  Erstellt ein neues Fenster mit einer leeren Einsatzplaner Instanz.

  \item[Öffnen \dots]
  Es wird ein Dialog geöffnet, mit dem eine \texttt{.ako}-Datei geöffnet werden kann.

  \item[Zuletzt Benutzt]
  Unter diesem Eintrag finden Sie die füf zuletzt verwendeten Dateien.
  Sie können direkt geöffnet werden.
  Ebenso können Sie bei Bedarf die Liste leeren.

  \item[Speichern]
  Die Datei wird an dem bisher bekannten Pfad gesichert, oder Sie werden nach einem Ort zum Sichern der Datei gefragt.

  \item[Sichern unter \dots]
  Sie können einen Ort auswählen, an dem die Datei gespeichert werden soll.

  Zum automatischen Sichern der Datei finden Sie weitere Informationen im Kapitel Einstellungen dieser Anleitung.

  \item[Stammdaten sichern unter \dots]
  Diese Funktion bietet die Möglichkeit, dass die unveränderlichen Daten exportiert werden; unter anderem Personaldaten und Einstellungen.
  Es werden keine Fahrtage oder Arbeitseinsätze gespeichert.
  Diese Funktion dupliziert sozusagen die aktuelle Datei und löscht dabei alle Fahrtage und Arbeitseinsätze.


  \item[Eigenschaften]
  Hier können Sie das Online-Tool aktivieren und konfigurieren.
  Weitere Informationen im entsprechenden Kapitel.

  \item[Schließen]
  Schließt das aktuelle Fenster.
  Wenn kein Fenster mehr geöffnet ist, wird das Programm beendet.

  \item[Exportieren \dots]
  Ruft die Exportfunktion auf.
  Weitere Funktionen in Kapitel \ref{export}.
\end{description}
