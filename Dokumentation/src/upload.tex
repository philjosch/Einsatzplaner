\chapter{Upload-Tool}
\section{Einsatzplaner}
Dieses Tool gibt ihnen die Möglichkeit die Listenansicht des Einsatzplan als PDF-Datei auf einen Webserver hochzuladen.
Dies kann manuell oder automatisch bei jedem Speichern geschehen.
Zur Konfiguration befindet sich im Menü ``Datei'' ein Punkt ``Eigenschaften''.
In dem sich öffnenden Menü finden Sie folgenden Abschnitt:

\begin{itemize}
  \item
  Mit dem ersten Haken wird das Tool aktiviert.
  \item
  Der zweite dient dazu einzustellen, ob die Datei automatisch hochgeladen wird, wenn die lokale Datei gespeichert wird.
  Diese Funktion ist nur verfügbar, wenn sie im Einstellung Menü des Programms aktiviert wurde.
  \item
  Unter Server, Pfad und ID geben Sie die Daten an, die Ihnen von ihrem Webmaster mitgeteilt wurden (http und https sind als Protokolle möglich).
  \item
  Mit dem Knopf wird getestet, ob ihre Eingaben korrekt sind, dafür wird eine Verbindung zum Webserver aufgebaut.
\end{itemize}

Die drei letzten Einstellungen sind nur relevant, wenn die Datei automatisch hochgeladen werden soll.
Damit keine Rückfragen beim Speichern auftreten, können Sie hier einstellen, in welchem Zeitraum die Aktivitäten liegen müssen, damit Sie eingeschlossen werden.
Ebenso können Sie auswählen, ob Arbeitseinsätze auch ausgegeben werden.

Wichtig:
Diese Einstellungen werden nicht verwendet, wenn Sie die Export-Funktion verwenden!
Weitere Informationen im entsprechenden Kapitel.



\section{Server-Tool}
Zur Installation des Tools auf einem Server benötigen Sie eine nicht veraltete Version von PHP.

Die benötigten Dateien werden unter macOS mit dem Programm ausgeliefert.
Alternativ können Sie auf der Webseite heruntergeladen werden.
Weitere Informationen im Kapitel ``Weitere Informationen''.

Das Tool besteht aus der Konfigurationsdatei \texttt{config.php} und dem eigentlichen Tool \texttt{load.php}.
Der Name der Datei \texttt{load.php} kann an die eigenen Bedürfnisse angepasst werden.
Der Name der Konfigurationsdatei darf nicht verändert werden.
Ebenso müssen beide Dateien im gleichen Ordner liegen.

Die Konfiguration des Tools ist in der Datei \texttt{config.php} beschrieben.
